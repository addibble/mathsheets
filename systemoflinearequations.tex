\documentclass[12pt]{article}
\usepackage{amsmath, amssymb}
\usepackage{geometry}
\geometry{margin=1in}
\title{Systems of Linear Equations Worksheet}
\author{}
\date{}

\begin{document}

\maketitle

\section*{Instructions}
Solve the following word problems by setting up a system of equations. Define your variables, write the system using a curly brace, and solve using substitution. Show all your work. Convert standard form to slope-intercept form where applicable.

\section*{Problem 1: The Tortoise and the Hare}
The Tortoise and the Hare are racing along a straight path. The Tortoise moves at a steady speed of 0.2 miles per minute and starts the race at time zero. The Hare runs at 1.6 miles per minute but starts 30 minutes later. At what time does the Hare catch up to the Tortoise?

\vspace{1em}
\noindent Let \( t \) be the time (in minutes) since the Tortoise started. Let \( d \) represent distance in miles.

\begin{align*}
\left\{
  \begin{array}{l}
    d = 0.2t \\ 
    d = 1.6(t - 30)
  \end{array}
\right.
\end{align*}

\vspace{1em}
\noindent\textbf{Slope-intercept form:} Already given in slope-intercept form \( d = mt + b \).

\newpage

\section*{Problem 2: Alien Currency Exchange}
On the planet Zog, two types of currency exist: glibs and zargs. Two tourists visit a market. One spends 3 glibs and 4 zargs for a laser hat costing 76 zolts. Another spends 5 glibs and 2 zargs on a jet-pack burrito costing 82 zolts. What is the value of one glib and one zarg in zolts?

\vspace{1em}
\noindent Let \( g \) be the value of one glib, and \( z \) the value of one zarg.

\begin{align*}
\left\{
  \begin{array}{l}
    3g + 4z = 76 \\ 
    5g + 2z = 82
  \end{array}
\right.
\end{align*}

\vspace{1em}
\noindent\textbf{Convert to slope-intercept form (solve for \( z \))}:
\begin{align*}
3g + 4z &= 76 & \Rightarrow 4z = -3g + 76 &\Rightarrow z = -\frac{3}{4}g + 19 \\ 
5g + 2z &= 82 & \Rightarrow 2z = -5g + 82 &\Rightarrow z = -\frac{5}{2}g + 41
\end{align*}

\newpage

\section*{Problem 3: The Ice Cream Conundrum}
A local creamery sells two sizes of ice cream cones. A small cone uses 1 scoop of vanilla and 2 scoops of chocolate. A large cone uses 3 scoops of vanilla and 1 scoop of chocolate. Yesterday, they sold 50 cones and used 90 scoops of ice cream in total. How many small and large cones were sold?

\vspace{1em}
\noindent Let \( s \) be the number of small cones and \( l \) be the number of large cones.

\begin{align*}
\left\{
  \begin{array}{l}
    s + l = 50 \\ 
    2s + 4l = 90
  \end{array}
\right.
\end{align*}

\vspace{1em}
\noindent\textbf{Convert to slope-intercept form (solve for \( s \))}:
\begin{align*}
s + l &= 50 & \Rightarrow s = -l + 50 \\ 
2s + 4l &= 90 & \Rightarrow s = -2l + 45
\end{align*}

\newpage

\section*{Problem 4: Rocket Ride}
A rocket ride starts at ground level and launches upward. Its height (in meters) after \( t \) seconds is given by the formula \( h = -5t^2 + 40t \). At the same time, a drone rises straight up from a nearby platform at a constant speed of 20 meters per second. At what time(s) do the rocket and the drone reach the same height?

\vspace{1em}
\noindent Let \( t \) be time in seconds. Let \( h \) be the height.

\begin{align*}
\left\{
  \begin{array}{l}
    h = -5t^2 + 40t \\ 
    h = 20t
  \end{array}
\right.
\end{align*}

\end{document}
