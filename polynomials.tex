\documentclass{article}
\usepackage[utf8]{inputenc} % Handle UTF-8 encoding
\usepackage{amsmath, amssymb, tikz, geometry, multicol, polynom}
\usepackage{pgfplots} % For plotting graphs
\pgfplotsset{compat=1.17}
\usetikzlibrary{calc}
\geometry{margin=0.25in}

\begin{document}

\begin{multicols}{2}

\section*{Algebra and Polynomials Cheat Sheet}

\subsection*{Polynomial Basics}
A polynomial in one variable \(x\) has the form:
\[
a_n x^n + a_{n-1} x^{n-1} + \cdots + a_1 x + a_0,
\]
with real or complex coefficients and \(a_n \neq 0\).

\textbf{Degree:} The highest exponent \(n\).

\textbf{Leading Coefficient:} The coefficient \(a_n\).

\textbf{Example:}
\[
4x^3 - 3x^2 + 2x - 1
\]
has degree 3 and leading coefficient 4.

\subsection*{Factoring Techniques}

\subsubsection*{Greatest Common Factor (GCF)}
Factor out the largest common factor from all terms.

\textbf{Example:}
\[
6x^3 + 9x^2 = 3x^2(2x + 3).
\]

\subsubsection*{Difference of Squares}
\[
a^2 - b^2 = (a+b)(a-b).
\]

\textbf{Example:}
\[
x^2 - 9 = (x+3)(x-3).
\]

\subsubsection*{Sum and Difference of Cubes}
\[
a^3 - b^3 = (a - b)(a^2 + ab + b^2),
\]
\[
a^3 + b^3 = (a + b)(a^2 - ab + b^2).
\]

\textbf{Example:}
\[
8x^3 - 27 = (2x - 3)(4x^2 + 6x + 9).
\]

\subsubsection*{Factoring Quadratics (Trinomials)}
For \(ax^2 + bx + c\):

- If \(a=1\):
\[
x^2 + bx + c = (x+m)(x+n), \quad m+n=b, \, mn=c.
\]

\textbf{Example:}
\[
x^2 + 5x + 6 = (x+2)(x+3).
\]

- If \(a \neq 1\) (AC Method):
1. Compute \(ac\).
2. Find \(m,n\) such that \(m+n=b\) and \(mn=ac\).
3. Rewrite and factor by grouping.

\textbf{Example:}
\[
2x^2+7x+3: \quad ac=6, \text{ find }(6,1).
\]
\[
2x^2+7x+3=2x^2+6x+x+3=2x(x+3)+(x+3)=(2x+1)(x+3).
\]

\subsubsection*{Perfect Square Trinomials}
\[
a^2 \pm 2ab + b^2 = (a \pm b)^2.
\]

\textbf{Example:}
\[
x^2+6x+9=(x+3)^2.
\]

\subsubsection*{Factoring by Grouping}
Used for four-term polynomials by grouping terms and factoring out common factors.

\textbf{Example:}
\[
x^3+3x^2+2x+6.
\]
Group:
\[
(x^3+3x^2)+(2x+6)=x^2(x+3)+2(x+3)=(x^2+2)(x+3).
\]

\subsubsection*{Sum of Squares and Complex Factoring}
\[
a^2+b^2 \text{ does not factor over the reals, but } a^2+b^2=(a+bi)(a-bi) \text{ over } \mathbb{C}.
\]

\textbf{Example:} 
\[
x^2+1=(x+i)(x-i).
\]

\textbf{Expanding $(a+b)^2$ with Imaginary Numbers:}
\[
(a+i)^2 = a^2 + 2ai + i^2 = a^2 + 2ai - 1.
\]

\subsection*{Quadratic Equations}

\subsubsection*{Standard Form}
\[
ax^2+bx+c=0.
\]

\subsubsection*{Quadratic Formula}
\[
x=\frac{-b \pm \sqrt{b^2-4ac}}{2a}.
\]

\textbf{Discriminant:} \(\Delta=b^2-4ac\)
- \(\Delta>0\): Two distinct real roots.
- \(\Delta=0\): One real double root.
- \(\Delta<0\): Two complex conjugate roots.

\textbf{Example:}
Solve \(x^2 -4x+3=0\):
\[
\Delta=16-12=4>0.
\]
\[
x=\frac{4\pm2}{2}=3 \text{ or }1.
\]

\subsubsection*{Completing the Square}
Convert \(ax^2+bx+c\) to vertex form:
1. If \(a \neq 1\), divide through by \(a\).
2. Move \(c/a\) to the other side.
3. Add \(\left(\frac{b}{2a}\right)^2\) to both sides.
4. Factor the perfect square.

\textbf{Example:}
\[
x^2+6x+5: \quad x^2+6x=-5.
\]
Add \((6/2)^2=9\):
\[
x^2+6x+9=4 \implies (x+3)^2=4.
\]

\subsection*{Complex Roots and Conjugates}
If \(a+bi\) is a root, \(a-bi\) is also a root.

\textbf{Example:}
For \(x^2+1=0\):
Roots: \(i\) and \(-i\).

\subsection*{Rational Roots and Rational Root Theorem}
For a polynomial \(a_nx^n+\dots+a_0\), if there is a rational root \(\frac{p}{q}\) (in lowest terms), then:
- \(p \mid a_0\) (p divides the constant term)
- \(q \mid a_n\) (q divides the leading coefficient)

\textbf{Example:}
Consider \(2x^3-3x^2+4x-6\). Possible rational roots are \(\pm 1, \pm 2, \pm 3, \pm 6\) divided by the leading coeff factors (1,2). So test \(\pm1,\pm2,\pm3,\pm6,\pm\frac{1}{2},\pm\frac{3}{2}\), etc.

If a certain rational root is known, you can use it to find missing coefficients.  
\textbf{Example:}
Suppose a polynomial \(2x^2+mx+3\) has a root \(\frac{3}{2}\). Substitute \(x=\frac{3}{2}\):
\[
2\left(\frac{3}{2}\right)^2 + m\left(\frac{3}{2}\right) + 3=0.
\]
\[
2\cdot\frac{9}{4}+\frac{3m}{2}+3=0 \implies \frac{9}{2}+\frac{3m}{2}+3=0.
\]
Multiply by 2:
\[
9+3m+6=0 \implies 3m+15=0 \implies m=-5.
\]

\subsection*{Sum and Product of Roots}

For \(ax^2+bx+c=0\):
- Sum of roots \(= -\frac{b}{a}\)
- Product of roots \(= \frac{c}{a}\)

\textbf{Example:}
\[
x^2-5x+6=0.
\]
Sum of roots = \(5\), product of roots = \(6\). If you know sum = 5 and product = 6, you can reconstruct the quadratic: \(x^2 - (sum)x + (product)=x^2-5x+6\).

For a cubic \(ax^3+bx^2+cx+d=0\):
- Sum of roots = \(-b/a\)
- Sum of product of roots taken two at a time = \(c/a\)
- Product of roots = \(-d/a\)

\columnbreak

\section*{Dividing Polynomials}

\subsection*{Long Division}
Use polynomial long division if \(\deg(\text{dividend}) \geq \deg(\text{divisor})\).

\textbf{Example:}
Divide \((2x^3+3x^2 - x +5)\) by \((x+2)\):
\[
\polyset{vars=x}
\polylongdiv{2x^3+3x^2 - x +5}{x+2}
\]
Result: Quotient \(= 2x^2 - x + 1\), Remainder \(= 3\).

\subsection*{Synthetic Division}
Faster method when dividing by \((x-r)\).

\textbf{Example:}
Divide \(2x^3-3x^2+4x-5\) by \(x-2\):

\[
\begin{array}{c|cccc}
2 & 2 & -3 & 4 & -5 \\
  &   & 4  & 2 & 12 \\
\hline
  & 2 & 1  & 6 & 7
\end{array}
\]

Quotient: \(2x^2+x+6\), Remainder: \(7\).

\subsection*{Partial Fraction Decomposition}
Express \(\frac{P(x)}{Q(x)}\) as a sum of simpler fractions after factoring \(Q(x)\).

\textbf{Example with Repeated Factor:}
\[
\frac{2x+3}{(x-1)^2}=\frac{A}{x-1}+\frac{B}{(x-1)^2}.
\]
Multiply through by \((x-1)^2\):
\[
2x+3=A(x-1)+B.
\]
Set \(x=1\):
\[
2(1)+3=5=B.
\]
For \(x=0\):
\[
3=A(-1)+5 \implies -A= -2 \implies A=2.
\]
Thus:
\[
\frac{2x+3}{(x-1)^2}=\frac{2}{x-1}+\frac{5}{(x-1)^2}.
\]

\section*{Asymptotes of Rational Functions}

\subsubsection*{Vertical Asymptotes}
Where denominator = 0 (and numerator \(\neq0\)).

\textbf{Example:}
\[
f(x)=\frac{x+2}{x^2-4}, \; x=\pm2 \text{ are vertical asymptotes.}
\]

\subsubsection*{Horizontal/Oblique Asymptotes}
Compare degrees of numerator (N) and denominator (D):
- \(\deg(N)<\deg(D)\): \(y=0\).
- \(\deg(N)=\deg(D)\): \(y=\frac{\text{leading coeff(N)}}{\text{leading coeff(D)}}\).
- \(\deg(N)>\deg(D)\): No horizontal asymptote; may have an oblique asymptote.

\textbf{Example:}
\[
f(x)=\frac{2x^2+3}{x^2-x}.
\]
Degrees equal:
\[
y=\frac{2}{1}=2.
\]

\section*{Graphing Polynomials and Roots}

\subsection*{Multiplicity of Roots}
- Odd multiplicity: graph crosses the x-axis.
- Even multiplicity: graph touches and turns at the x-axis.

\textbf{Example:}
\[
f(x)=(x+2)(x-1)^2
\]
Root \(-2\) (odd): crosses.  
Root \(1\) (even): touches and turns.

\subsection*{End Behavior}
- Even degree, positive leading coeff: Up, Up.
- Even degree, negative leading coeff: Down, Down.
- Odd degree, positive leading coeff: Down, Up.
- Odd degree, negative leading coeff: Up, Down.

\subsection*{Intermediate Value Theorem}
If \(f(a)\) and \(f(b)\) have opposite signs, there's at least one root in \((a,b)\).

\section*{Parabolas}
A parabola is the graph of a quadratic function.

\textbf{Standard Form:}
\[
y=ax^2+bx+c.
\]

\textbf{Vertex Form:}
\[
y=a(x-h)^2 + k,
\]
where \((h,k)\) is the vertex and the axis of symmetry is \(x=h\).

To find the vertex from the standard form, use completing the square or \(-\frac{b}{2a}\):
\[
h = -\frac{b}{2a}, \quad k = f(h).
\]

\textbf{Example:}
\[
y=2x^2+4x+1.
\]
Find vertex:
\[
h=-\frac{4}{2\cdot2}=-1; \; k=2(-1)^2+4(-1)+1=2-4+1=-1.
\]
Vertex: \((-1,-1)\), so
\[
y=2(x+1)^2 -1.
\]

Parabolas open upwards if \(a>0\) and downwards if \(a<0\).
\subsection*{Hyperbolas}

A hyperbola is the set of all points \((x,y)\) in the plane such that the *difference* of their distances to two fixed points (the foci) is constant. 

\subsubsection*{Standard Forms}

Depending on whether the transverse axis is horizontal or vertical, the standard form of a hyperbola centered at \((h,k)\) can be written as:

\[
\frac{(x - h)^2}{a^2} - \frac{(y - k)^2}{b^2} = 1 
\quad\text{(horizontal transverse axis)},
\]
or
\[
\frac{(y - k)^2}{b^2} - \frac{(x - h)^2}{a^2} = 1 
\quad\text{(vertical transverse axis)}.
\]

Here:
- \((h,k)\) is the center of the hyperbola.
- \(a\) is the semi-major axis (distance from center to each vertex).
- \(b\) relates to the “co-vertices” and the asymptotes.
- The asymptotes for the hyperbola 
  \(\tfrac{(x-h)^2}{a^2} - \tfrac{(y-k)^2}{b^2} = 1\) 
  are given by 
  \[
    y - k \;=\; \pm \frac{b}{a} \,(x - h).
  \]
- The foci are located \(c\) units from the center, where \(c^2 = a^2 + b^2\).

\subsubsection*{Getting a Hyperbola into Standard Form from Polynomial Form}
Often, a hyperbola starts in the form of a general quadratic equation:
\[
Ax^2 + Bxy + Cy^2 + Dx + Ey + F = 0,
\]
with certain conditions on \(A\), \(B\), and \(C\) (specifically, for a hyperbola you typically have \(B^2 - 4AC > 0\) when \(B=0\); or more generally, the discriminant indicates a hyperbola).

\textbf{Steps to rewrite into standard form:}
1. \textbf{Group the \(x\)-terms and \(y\)-terms}:  
   Separate the equation into something like
   \[
   Ax^2 + Dx \;+\; Cy^2 + Ey \;=\; -F.
   \]
2. \textbf{Factor out the coefficients of the squared terms}:
   \[
   A\bigl(x^2 + \tfrac{D}{A}x\bigr) \;+\; C\bigl(y^2 + \tfrac{E}{C}y\bigr) = -F.
   \]
3. \textbf{Complete the square} for the \(x\)-group and the \(y\)-group separately:
   \[
   A\Bigl(x^2 + \tfrac{D}{A}x + \dots\Bigr) - A\bigl(\dots\bigr)
   + C\Bigl(y^2 + \tfrac{E}{C}y + \dots\Bigr) - C\bigl(\dots\bigr)
   = -F.
   \]
   The “\(\dots\)” refers to the term you add (and subtract) to complete the square. 
4. \textbf{Rearrange and combine constants} to isolate perfect square expressions in \(x\) and \(y\). 
5. \textbf{Divide through} by whatever constant makes the right side \(=1\) (or \(-1\), then multiply by \(-1\) if needed). 
   You’ll end up with one positive term and one negative term (equal to \(1\)), matching one of the standard forms for a hyperbola.

\textbf{Example:}  
Rewrite \(4x^2 - 9y^2 + 8x + 18y = 77\) in standard form.
\begin{enumerate}
\item Group \(x\)-terms and \(y\)-terms:
  \[
    4x^2 + 8x \;-\; 9y^2 + 18y = 77.
  \]
\item Factor out the coefficients of squared terms:
  \[
    4(x^2 + 2x) \;-\; 9(y^2 - 2y) = 77.
  \]
\item Complete the square for each group:

  For \(x^2 + 2x\): 
  \[
    x^2 + 2x + 1 - 1 = (x+1)^2 - 1.
  \]

  For \(y^2 - 2y\):
  \[
    y^2 - 2y + 1 - 1 = (y-1)^2 - 1.
  \]

  Substituting these back,
  \[
    4\bigl[(x+1)^2 - 1\bigr]
    \;-\; 9\bigl[(y-1)^2 - 1\bigr] 
    = 77.
  \]
  Distribute:
  \[
    4(x+1)^2 - 4 
    \;-\; 9(y-1)^2 + 9 
    = 77.
  \]
  Combine constants:
  \[
    4(x+1)^2 - 9(y-1)^2 + (9 - 4) = 77
    \;\;\implies\;\;
    4(x+1)^2 - 9(y-1)^2 + 5 = 77.
  \]
  \[
    4(x+1)^2 - 9(y-1)^2 = 72.
  \]

\item Divide through by 72:
  \[
    \frac{4(x+1)^2}{72} \;-\; \frac{9(y-1)^2}{72} = 1,
  \]
  Simplify:
  \[
    \frac{(x+1)^2}{18} \;-\; \frac{(y-1)^2}{8} = 1.
  \]
\item This is in the standard hyperbola form:
  \[
    \frac{(x - (-1))^2}{(\sqrt{18})^2} 
    \;-\; 
    \frac{(y - 1)^2}{(\sqrt{8})^2} 
    = 1.
  \]
\end{enumerate}

\end{multicols}

\end{document}
