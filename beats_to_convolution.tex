% !TEX TS-program = lualatex
\documentclass[12pt]{article}

% -------------------------------------------------
% Packages
% -------------------------------------------------
\usepackage{fontspec}                 % modern Unicode fonts
\usepackage{amsmath,amssymb,physics}  % maths
\usepackage{geometry}                 % page layout
\usepackage{graphicx}                 % images
\usepackage{hyperref}                 % links
\usepackage{listings}                 % nicely-formatted code
\usepackage{xcolor}                   % colours for listings

\geometry{margin=1in}

% -------------------------------------------------
% Listings setup
% -------------------------------------------------
\lstset{
  language        = Python,
  basicstyle      = \ttfamily\small,
  commentstyle    = \color{gray},
  keywordstyle    = \color{blue},
  stringstyle     = \color{teal},
  breaklines      = true,
  breakautoindent = true,
  showstringspaces= false,
  columns         = fullflexible
}

% -------------------------------------------------
\title{From Beats to Diffraction\\\large A Journey Through Waves}
\date{\today}
% -------------------------------------------------
\begin{document}
\maketitle
\parindent 0pt
\parskip   6pt

%==================================================
\section*{Prologue}
We will travel through sound, light, and water the way early scientists did—asking \emph{why?} at every turn, inventing just \textbf{one new piece of maths} at a time, and checking our ideas with code you can run in Google~Colab.

%==================================================
\section{The First Mystery: Why Two Notes Throb (Audio Beats)}
\subsection*{A Little Scene from 1787}
Picture \textit{Ernst Chladni} striking two tuning-forks that disagree by the tiniest smidge. The air grows loud–soft–loud–soft like a giant breathing. Chladni wants to explain that pulse. Radio pioneer \textit{Reginald Fessenden} will later harness the same trick (\textasciitilde1900) to ride news and music on invisible waves.

\subsection*{What We Are About to Do}
We will \textbf{multiply} a fast wiggle (the \emph{carrier}) by a slow wiggle (the \emph{envelope}). When the envelope is high, the carrier grows; when it is low, the carrier shrinks. That arithmetic trick is called \emph{amplitude modulation}.

\subsection*{The One New Formula}
\begin{equation}
  y(t)=A\,\sin(2\pi f_{c}t)\,\cos(2\pi f_{e}t)
\end{equation}

\textbf{How it works:}
\begin{itemize}
  \item \(A\): a constant that multiplies the combined wave to control overall loudness.
  \item \(t\): the running clock on your stopwatch.
  \item \(f_{c}\): carrier frequency—how many tiny wiggles per second (440\,Hz is the musical A).
  \item \(f_{e}\): envelope frequency—how many loud–soft cycles per second.
  \item \(\pi\): lower-case Greek pi (first printed by \textit{William Jones}, 1706).
\end{itemize}

\subsection*{Workable Examples}
\begin{enumerate}
  \item \textbf{Finding loudness at two instants.}  With \(A=1\), \(f_{c}=440\,\text{Hz}\) and \(f_{e}=5\,\text{Hz}\), compute \(y(0.002\,\text{s})\) and \(y(0.004\,\text{s})\) using the formula.
  \item \textbf{Open question.}  If we double the envelope frequency to 10\,Hz, how does the time between loud peaks change?
\end{enumerate}

\subsection*{Python Check-up (Colab)}
\begin{lstlisting}[caption=Beats in Python,label=lst:beats]
import numpy as np, matplotlib.pyplot as plt

A, fc, fe = 1, 440, 5
t = np.linspace(0, 0.5, 20_000)
y = A*np.sin(2*np.pi*fc*t) * np.cos(2*np.pi*fe*t)

plt.plot(t*1e3, y)
plt.xlabel('Time (ms)')
plt.ylabel('Amplitude')
plt.title('Audio Beat Wave')
plt.show()
\end{lstlisting}

%==================================================
\section{The Bell-Shaped Secret Window (Gaussian)}
\subsection*{Gauss and the Star That Wouldn’t Sit Still (1809)}
Astronomer \textit{Carl Friedrich Gauss} tried to guess where the dwarf planet Ceres had gone after it slipped behind the Sun. He needed a maths tool that gives \emph{full credit to the middle} and \emph{little credit to the edges}: the famous bell curve.

\subsection*{What We Are About to Do}
We will swap the cosine envelope for a \textbf{smooth bell} that starts small, swells gently, and falls away without a bump, producing a \emph{single burst} instead of endless throbs.

\subsection*{The One New Formula}
\begin{equation}
  G(t)=\exp\!\bigl(-t^{2}/(2\sigma^{2})\bigr)
\end{equation}

\textbf{How it works:}
\begin{itemize}
  \item Inside the exponent, \(t^{2}\) grows fast as you walk away from zero; the minus sign makes \(G(t)\) shrink.
  \item \(\sigma\) (sigma, lower-case; adopted for standard deviation by \textit{Karl Pearson}, 1894) sets how wide the hill is—larger \(\sigma\) makes a wider bell.
\end{itemize}

\subsection*{Workable Examples}
\begin{enumerate}
  \item \textbf{Half-second hill.}  With \(\sigma=0.1\,\text{s}\) find \(G(0)\) and \(G(0.1)\).
  \item \textbf{Try widening the hill.}  Double \(\sigma\) to 0.2\,s. How tall is \(G(0.1)\) now?
\end{enumerate}

\subsection*{Python Sketch}
\begin{lstlisting}[caption=Gaussian envelope,label=lst:gaussian]
import numpy as np, matplotlib.pyplot as plt

sigma = 0.1
t = np.linspace(-0.5, 0.5, 1_000)
G = np.exp(-(t**2) / (2*sigma**2))

plt.plot(t, G)
plt.title('Gaussian Envelope')
plt.xlabel('t (s)')
plt.ylabel('G(t)')
plt.show()
\end{lstlisting}

%==================================================
\section{A Single Audible Ping (Gaussian-Windowed Sine)}
\subsection*{Gabor’s 1946 Thought Experiment}
Engineer \textit{Dennis Gabor} imagined cutting a sine wave out of time with a bell window so he could paint pictures of sound in both time \emph{and} frequency. That idea won him a Nobel Prize and later gave birth to MP3s.

\subsection*{What We Are About to Do}
Glue the hill from Section~2 on top of the sine from Section~1. When the hill is high, the tone is loud; when low, silence.

\subsection*{The One New Formula}
\begin{equation}
  y(t)=A\,e^{-t^{2}/(2\sigma^{2})}\,\sin\bigl(2\pi f_{c}t\bigr)
\end{equation}

\subsection*{Workable Examples}
\begin{enumerate}
  \item \textbf{Quick ping.}  With \(f_{c}=440\,\text{Hz}\) and \(\sigma=0.01\,\text{s}\), count the number of crest-to-crest oscillations under the bell. (Hint: audible span \(\approx6\sigma\).)
  \item \textbf{Your turn.}  Make the bell five times wider. How many oscillations now?
\end{enumerate}

\subsection*{Python — Listen with Your Eyes}
\begin{lstlisting}[caption=Gaussian-windowed tone,label=lst:ping]
import numpy as np, matplotlib.pyplot as plt

A, fc, sigma = 1, 440, 0.01
t = np.linspace(-0.05, 0.05, 4_000)
pulse = A*np.exp(-(t**2)/(2*sigma**2)) * np.sin(2*np.pi*fc*t)

plt.plot(t*1e3, pulse)
plt.xlabel('Time (ms)')
plt.title('Gaussian-Windowed Audio Pulse')
plt.show()
\end{lstlisting}

%==================================================
\section{Slowing the Music for Water (Scaling Frequency)}
\subsection*{George Airy Watches Ocean Swells (1845)}
On a windy day the sea looks nothing like a violin string. \textit{Sir George Airy} noticed that water waves unfold in slow, heavy packets—perfect cousins to our audio ping but with far fewer wiggles.

\subsection*{What We Are About to Do}
Keep the same bell but \textbf{stretch time} so the sine hardly has time to wiggle. The trick: divide the frequency by a big number \(N\).

\subsection*{The One New Formula}
\begin{equation}
  f_{w}=\frac{f_{c}}{N}, \qquad N\gg1
\end{equation}

\subsection*{Workable Examples}
\begin{enumerate}
  \item \textbf{Make 440 Hz crawl.}  With \(N=100\), compute \(f_{w}\) and sketch a water pulse under \(\sigma=0.5\) s.
  \item \textbf{Challenge.}  What if \(N=500\)?
\end{enumerate}

\subsection*{Python Ripple}
\begin{lstlisting}[caption=Slow ripple,label=lst:ripple]
import numpy as np, matplotlib.pyplot as plt

A, fc, N, sigma = 1, 440, 100, 0.5
fw = fc / N
t = np.linspace(-4, 4, 8_000)
water = A*np.exp(-(t**2)/(2*sigma**2)) * np.sin(2*np.pi*fw*t)

plt.plot(t, water)
plt.title('Gaussian-Windowed Water Pulse')
plt.xlabel('t (s)')
plt.show()
\end{lstlisting}

%==================================================
\section{From Line to Circle — The Pebble Splash (2-D Ripple)}
\subsection*{Poisson \& Fresnel Draw Ripples (1818)}
Working for Napoleon’s map-makers, \textit{Siméon Poisson} solved how a single splash spreads; \textit{Augustin-Jean Fresnel} soon borrowed the idea for light.

\subsection*{What We Are About to Do}
Drop our Section~3 pulse at the origin of a pond. The crest now forms a \textbf{circle}. We introduce the word \emph{wavefront}: the thin line marking everywhere the crest has reached.

\subsection*{The One New Formula}
\begin{equation}
  u(r,t)=A\,e^{-(r-vt)^{2}/(2\sigma^{2})}\,\cos\bigl(k(r-vt)\bigr)
\end{equation}

\textbf{How it works:}
\begin{itemize}
  \item \(r=\sqrt{x^{2}+y^{2}}\) is distance from the splash.
  \item \(v\) is wave speed.
  \item \(k=2\pi/\lambda\) is the wave-number (\textit{Scott Russell}, 1844).
  \item \(\lambda\) is wavelength (lower-case lambda; \textit{William Thomson}, 1850).
\end{itemize}

\subsection*{Workable Examples}
\begin{enumerate}
  \item \textbf{Pebble numbers.}  With \(v=0.3\) m/s, \(\lambda=0.1\) m, \(\sigma=0.05\) m, find the crest radius at \(t=1\) s and compute \(k\).
  \item \textbf{Double \(\lambda\).}  How does ring spacing change?
\end{enumerate}

\subsection*{Python Pond Picture}
\begin{lstlisting}[caption=Circular ripple,label=lst:pond]
import numpy as np, matplotlib.pyplot as plt

A, v, lam, sigma, t = 1, 0.3, 0.1, 0.05, 1.0
k = 2*np.pi / lam

x = y = np.linspace(-0.5, 0.5, 400)
X, Y = np.meshgrid(x, y)
r = np.sqrt(X**2 + Y**2)
U = A*np.exp(-((r - v*t)**2)/(2*sigma**2)) * np.cos(k*(r - v*t))

plt.imshow(U, extent=[-0.5, 0.5, -0.5, 0.5], origin='lower', cmap='RdBu')
plt.colorbar(label='Amplitude')
plt.title('Circular Ripple at t = 1 s')
plt.xlabel('x (m)')
plt.ylabel('y (m)')
plt.show()
\end{lstlisting}

%==================================================
\section{From 1-D to 2-D – Circular Ripple Packet}

\subsection*{What We Are About to Do}
We let the Gaussian-windowed tone from Section 4 explode outward in every direction.  
Each crest is now a \textbf{circle} whose radius grows at the wave-speed \(v\).  
The bell envelope still keeps the packet finite, so it travels as a neat, moving hill on the pond.

\subsection*{The One New Formula}
\begin{equation}
U(r,t)=A\,\exp\!\Bigl[-\frac{(r-vt)^2}{2\sigma^{2}}\Bigr]\,
\cos\bigl(k(r-vt)\bigr)
\end{equation}

\subsection*{How it Works}
\begin{itemize}
  \item \(r=\sqrt{x^{2}+y^{2}}\) is how far you are from the splash point.
  \item The bell centre sits at \(r=vt\); that term makes the packet \emph{move} without changing shape.
  \item The exponential provides the smooth bell.  Width \(\sigma\) decides how fat the packet is.
  \item \(k=2\pi/\lambda\) sets ripple spacing; bigger \(k\) means more ripples per metre.
  \item \(A\) is a master volume knob for the whole field.
\end{itemize}

\subsection*{Workable Examples}
\begin{enumerate}
  \item \textbf{Surface ripple.}  With \(v=0.30\;\text{m s}^{-1}\), \(\lambda=0.15\;\text{m}\), \(\sigma=0.20\;\text{m}\), find the crest radius at \(t=2\;\text{s}\).
  \item \textbf{Slower packet.}  Keep \(\lambda,\sigma\) but set \(v=0.10\;\text{m s}^{-1}\). Where is the crest at \(t=2\;\text{s}\)?
\end{enumerate}

\subsection*{Python/Colab Verification}
\begin{lstlisting}[caption=Circular packet,label=lst:packet]
import numpy as np, matplotlib.pyplot as plt

x = y = np.linspace(-1, 1, 400)
X, Y = np.meshgrid(x, y)
r = np.sqrt(X**2 + Y**2)

v, lam, sigma, t = 0.30, 0.15, 0.20, 2.0
k = 2*np.pi / lam
U = np.exp(-(r - v*t)**2 / (2*sigma**2)) * np.cos(k * (r - v*t))

plt.imshow(U, cmap='RdBu', extent=[-1, 1, -1, 1])
plt.colorbar()
plt.title('Circular Ripple Packet')
plt.show()
\end{lstlisting}

%==================================================
\section{Straight-Edge Impulse \& Convolution of Many Point Sources}

\subsection*{What We Are About to Do}
Instead of a single splash, we strike an entire straight board on the water.  
We model the board by adding \emph{infinitely many} little circular packets—one for each point along the edge—and summing them up with an integral.

\subsection*{The One New Formula}
\begin{equation}
U_{\text{line}}(x,y,t)=\int_{-L/2}^{L/2}
U\!\bigl(r',t\bigr)\,\dd x', \qquad
r'=\sqrt{(x-x')^{2}+y^{2}}
\end{equation}

\subsection*{How it Works}
\begin{itemize}
  \item Each \(x'\) along the board launches a packet \(U(r',t)\).
  \item The distance \(r'\) from that source to the observation point \((x,y)\) sets the packet’s delay.
  \item The integral \(\int\dd x'\) adds all these contributions, building the full wavefront.
  \item Board length \(L\) controls how wide the summed pulse is: long boards make flatter, broader fronts.
\end{itemize}

\subsection*{Workable Examples}
\begin{enumerate}
  \item \textbf{Long board.}  \(L=1.0\;\text{m}\), \(v=0.30\;\text{m s}^{-1}\), \(\sigma=0.20\;\text{m}\), \(t=1\;\text{s}\) — show the nearly flat crest at \(y=0.5\;\text{m}\).
  \item \textbf{Short paddle.}  \(L=0.20\;\text{m}\) — observe the much stronger curvature.
\end{enumerate}

\subsection*{Python/Colab Verification}
\begin{lstlisting}[caption=Line-source superposition,label=lst:line]
import numpy as np, matplotlib.pyplot as plt

v, lam, sigma = 0.30, 0.15, 0.20
k = 2*np.pi / lam
L, t, y_obs = 1.0, 1.0, 0.5

x_obs = np.linspace(-2, 2, 400)

def point_packet(r):
    return np.exp(-(r - v*t)**2/(2*sigma**2)) * np.cos(k*(r - v*t))

U_line = []
for x in x_obs:
    xs = np.linspace(-L/2, L/2, 400)
    r = np.sqrt((x - xs)**2 + y_obs**2)
    U_line.append(np.trapz(point_packet(r), xs))

plt.plot(x_obs, U_line)
plt.title('Wavefront from Straight Edge')
plt.xlabel('x (m)')
plt.show()
\end{lstlisting}

%==================================================
\section{Obstacle \& Re-Emission – Diffraction at a Slit}

\subsection*{What We Are About to Do}
We shine our wave through a narrow opening and ask how bright the far-field pattern is at each angle.  
Mathematically, that means taking the Fourier transform of a rectangular aperture—classical single-slit diffraction.

\subsection*{The One New Formula}
\begin{equation}
I(\theta)=I_{0}\Bigl[\tfrac{\sin\beta}{\beta}\Bigr]^{2},
\qquad
\beta=\frac{\pi a}{\lambda}\sin\theta
\end{equation}

\subsection*{How it Works}
\begin{itemize}
  \item \(a\) is slit width; narrower \(a\) spreads the pattern wider.
  \item \(\theta\) measures the observation angle from the central axis.
  \item \(\beta\) is a phase term: each path through the slit gains a phase proportional to \(\sin\theta\).
  \item The squared \(\sin\beta/\beta\) factor makes a bright central maximum and evenly spaced dark minima where \(\beta=n\pi\).
  \item \(I_{0}\) simply scales the whole intensity curve.
\end{itemize}

\subsection*{Workable Examples}
\begin{enumerate}
  \item \textbf{Red laser.}  \(\lambda=650\,\text{nm}\), \(a=0.10\,\text{mm}\).  First dark at \(\sin\theta\approx\lambda/a\approx6.5\times10^{-3}\). Calculate \theta.
  \item \textbf{Narrower slit.}  \(a=0.05\,\text{mm}\) — the first-dark angle changes to what \theta?
\end{enumerate}

\subsection*{Python/Colab Verification}
\begin{lstlisting}[caption=Single-slit diffraction,label=lst:slit]
import numpy as np, matplotlib.pyplot as plt

lam, a = 532e-9, 0.10e-3
theta = np.linspace(-5e-3, 5e-3, 2_000)
beta  = np.pi * a * np.sin(theta) / lam
I     = (np.sinc(beta / np.pi))**2

plt.plot(np.degrees(theta), I)
plt.xlabel(r'$\theta$ (deg)')
plt.ylabel('Relative Intensity')
plt.title('Single-Slit Fraunhofer Diffraction')
plt.show()
\end{lstlisting}

\end{document}
