\documentclass{article}
\usepackage[utf8]{inputenc}
\usepackage{amsmath, amssymb, tikz, geometry, multicol}
\geometry{margin=0.25in}

\begin{document}

\begin{multicols}{2}

\section*{Differentiation}

\subsection*{Definition}

The derivative of a function \( f(x) \) at a point \( x \) is defined as:
\[
f'(x) = \lim_{h \to 0} \frac{f(x+h) - f(x)}{h}
\]

\subsection*{Power Rule}
\[
f(x) = x^n \Rightarrow f'(x) = nx^{n-1}
\]
Example:
\[
f(x) = x^3 \Rightarrow f'(x) = 3x^2
\]

\subsection*{Constant Rule}
\[
f(x) = c \Rightarrow f'(x) = 0
\]
Example:
\[
f(x) = 2 \Rightarrow f'(x) = 0
\]

\subsection*{Basic Derivatives}
\[
\frac{d}{dx} e^x = e^x
\]
\[
\frac{d}{dx} x^n = n x^{n-1}
\]
\[
\frac{d}{dx} \ln x = \frac{1}{x} \quad \text{for } x > 0
\]

\subsection*{Derivative of Exponential Functions}
For \( f(x) = a^x \), where \( a > 0 \) and \( a \neq 1 \):
\[
\frac{d}{dx} a^x = a^x \ln a
\]
Example:
\[
\frac{d}{dx} 2^x = 2^x \ln 2
\]

\subsection*{Derivative of Logarithmic Functions}
For \( f(x) = \log_a x \), where \( a > 0 \) and \( a \neq 1 \):
\[
\frac{d}{dx} \log_a x = \frac{1}{x \ln a}
\]
Example:
\[
\frac{d}{dx} \log_2 x = \frac{1}{x \ln 2}
\]

\subsection*{Derivatives of Trigonometric Functions}
\[
\frac{d}{dx} \sin x = \cos x
\]
\[
\frac{d}{dx} \cos x = -\sin x
\]
\[
\frac{d}{dx} \tan x = \sec^2 x
\]
\[
\frac{d}{dx} \cot x = -\csc^2 x
\]
\[
\frac{d}{dx} \sec x = \sec x \tan x
\]
\[
\frac{d}{dx} \csc x = -\csc x \cot x
\]
Example:
\[
\frac{d}{dx} \tan x = \sec^2 x
\]
\[
\frac{d}{dx} \cot x = -\csc^2 x
\]

\subsection*{Product Rule}
If \( f(x) = u(x) \cdot v(x) \):
\[
f'(x) = u'v + uv'
\]

\subsection*{Quotient Rule}
If \( f(x) = \dfrac{u(x)}{v(x)} \):
\[
f'(x) = \frac{u'v - uv'}{v^2}
\]

\subsection*{Chain Rule}
If \( f(x) = h(g(x)) \), then:
\[
f'(x) = h'(g(x)) \cdot g'(x)
\]
Example:
\[
f(x) = (2x^3 + x)^4
\]
Then:
\[
f'(x) = 4(2x^3 + x)^3 \cdot (6x^2 + 1)
\]

\columnbreak

\section*{Integration}
\subsection*{Definition of Integration}

The definite integral of a function \( f(x) \) over the interval \([a, b]\) is defined as the limit of a Riemann sum:

\[
\int_{a}^{b} f(x) \, dx = \lim_{n \to \infty} \sum_{i=1}^{n} f(x_i^*) \Delta x
\]

where:
- \( \Delta x = \dfrac{b - a}{n} \) is the width of each subinterval,
- \( x_i^* \) is a sample point in the \( i \)-th subinterval \([x_{i-1}, x_i]\).

The integral represents the accumulation of the quantity \( f(x) \) over the interval \([a, b]\).

Alternatively, the indefinite integral (antiderivative) of a function \( f(x) \) is a function \( F(x) \) such that:

\[
\int f(x) \, dx = F(x) + C
\]

where:
- \( F'(x) = f(x) \),
- \( C \) is the constant of integration.

\subsection*{Power Rule}
\[
\int x^n \, dx = \frac{x^{n+1}}{n+1} + C \quad \text{(for } n \neq -1\text{)}
\]

\subsection*{Constant Rule}
\[
\int c \, dx = cx + C
\]
Example:
\[
\int 5x^2 \, dx = 5 \int x^2 \, dx = 5 \cdot \frac{x^3}{3} = \frac{5x^3}{3} + C
\]

\subsection*{Sum Rule}
\[
\int [f(x) + g(x)] \, dx = \int f(x) \, dx + \int g(x) \, dx
\]

\subsection*{Basic Integrals}
\[
\int e^x \, dx = e^x + C
\]
\[
\int a^x \, dx = \frac{a^x}{\ln a} + C \quad \text{(for } a > 0,\, a \neq 1\text{)}
\]
Example:
\[
\int 2^x \, dx = \frac{2^x}{\ln 2} + C
\]
\[
\int \frac{1}{x} \, dx = \ln |x| + C
\]
\[
\int \ln x \, dx = x \ln x - x + C
\]
\[
\int \log_a x \, dx = x \log_a x - \frac{x}{\ln a} + C
\]
Example:
\[
\int \log_2 x \, dx = x \log_2 x - \frac{x}{\ln 2} + C
\]

\subsection*{Integrals of Trigonometric Functions}
\[
\int \sin x \, dx = -\cos x + C
\]
\[
\int \cos x \, dx = \sin x + C
\]
\[
\int \tan x \, dx = -\ln |\cos x| + C
\]
\[
\int \cot x \, dx = \ln |\sin x| + C
\]
\[
\int \sec x \, dx = \ln |\sec x + \tan x| + C
\]
\[
\int \csc x \, dx = -\ln |\csc x + \cot x| + C
\]
\[
\int \sec^2 x \, dx = \tan x + C
\]
\[
\int \csc^2 x \, dx = -\cot x + C
\]
\[
\int \sec x \tan x \, dx = \sec x + C
\]
\[
\int \csc x \cot x \, dx = -\csc x + C
\]

\subsection*{Example Calculations}
1. \[
\int x^2 \, dx = \frac{x^3}{3} + C
\]
2. \[
\int 3^x \, dx = \frac{3^x}{\ln 3} + C
\]
3. \[
\int \ln x \, dx = x \ln x - x + C
\]
4. \[
\int \frac{1}{x \ln a} \, dx = \log_a x + C
\]
5. \[
\int \sec^2 x \, dx = \tan x + C
\]
6. \[
\int \tan x \, dx = -\ln |\cos x| + C
\]

\end{multicols}

\end{document}
