\documentclass{article}
\usepackage[utf8]{inputenc}
\usepackage{amsmath, amssymb, tikz, geometry, multicol}
\geometry{margin=0.25in}

\begin{document}

\begin{multicols}{2}

\section*{Differentiation}

\subsection*{Definition}

The derivative of a function \( f(x) \) at a point \( x \) is defined as:
\[
f'(x) = \lim_{h \to 0} \frac{f(x+h) - f(x)}{h}
\]

\subsection*{Power Rule}
\[
f(x) = x^n \Rightarrow f'(x) = nx^{n-1}
\]
Example:
\[
f(x) = x^3 \Rightarrow f'(x) = 3x^2
\]

\subsection*{Constant Rule}
\[
f(x) = c \Rightarrow f'(x) = 0
\]
Example:
\[
f(x) = 2 \Rightarrow f'(x) = 0
\]

\subsection*{Basic Derivatives}
\[
\frac{d}{dx} e^x = e^x
\]
\[
\frac{d}{dx} x^n = n x^{n-1}
\]
\[
\frac{d}{dx} \ln x = \frac{1}{x} \quad \text{for } x > 0
\]

\subsection*{Derivative of Exponential Functions}
For \( f(x) = a^x \), where \( a > 0 \) and \( a \neq 1 \):
\[
\frac{d}{dx} a^x = a^x \ln a
\]
Example:
\[
\frac{d}{dx} 2^x = 2^x \ln 2
\]

\subsection*{Derivative of Logarithmic Functions}
For \( f(x) = \log_a x \), where \( a > 0 \) and \( a \neq 1 \):
\[
\frac{d}{dx} \log_a x = \frac{1}{x \ln a}
\]
Example:
\[
\frac{d}{dx} \log_2 x = \frac{1}{x \ln 2}
\]

\subsection*{Derivatives of Trigonometric Functions}
\[
\frac{d}{dx} \sin x = \cos x
\]
\[
\frac{d}{dx} \cos x = -\sin x
\]
\[
\frac{d}{dx} \tan x = \sec^2 x
\]
\[
\frac{d}{dx} \cot x = -\csc^2 x
\]
\[
\frac{d}{dx} \sec x = \sec x \tan x
\]
\[
\frac{d}{dx} \csc x = -\csc x \cot x
\]

\subsection*{Derivatives of Inverse Trigonometric Functions}
\[
\frac{d}{dx} \arcsin x = \frac{1}{\sqrt{1 - x^2}}
\]
\[
\frac{d}{dx} \arccos x = -\frac{1}{\sqrt{1 - x^2}}
\]
\[
\frac{d}{dx} \arctan x = \frac{1}{1 + x^2}
\]
\[
\frac{d}{dx} \operatorname{arccot} x = -\frac{1}{1 + x^2}
\]
\[
\frac{d}{dx} \operatorname{arcsec} x = \frac{1}{|x|\sqrt{x^2 - 1}}
\]
\[
\frac{d}{dx} \operatorname{arccsc} x = -\frac{1}{|x|\sqrt{x^2 - 1}}
\]

\subsection*{Product Rule}
If \( f(x) = u(x) \cdot v(x) \):
\[
f'(x) = u'v + uv'
\]

\subsection*{Quotient Rule}
If \( f(x) = \dfrac{u(x)}{v(x)} \):
\[
f'(x) = \frac{u'v - uv'}{v^2}
\]

\subsection*{Chain Rule}
If \( f(x) = h(g(x)) \), then:
\[
f'(x) = h'(g(x)) \cdot g'(x)
\]
Example:
\[
f(x) = (2x^3 + x)^4
\]
Then:
\[
f'(x) = 4(2x^3 + x)^3 \cdot (6x^2 + 1)
\]

\subsection*{Implicit Differentiation}

Implicit differentiation is used to find the derivative of a function when it is not explicitly solved for \(y\) in terms of \(x\). For example, consider the equation:
\[
x^2 + y^2 = 1
\]
To differentiate implicitly:
\begin{enumerate}
    \item Differentiate both sides of the equation with respect to \(x\), treating \(y\) as a function of \(x\) (\(y = y(x)\)).
    \item Apply the chain rule to terms involving \(y\).
\end{enumerate}
\textbf{Steps:}
\begin{itemize}
    \item Differentiate \(x^2 + y^2 = 1\):
    \[
    \frac{d}{dx}(x^2) + \frac{d}{dx}(y^2) = \frac{d}{dx}(1)
    \]
    \item Simplify:
    \[
    2x + 2y \frac{dy}{dx} = 0
    \]
    \item Solve for \(\frac{dy}{dx}\):
    \[
    \frac{dy}{dx} = -\frac{x}{y}
    \]
\end{itemize}

\textbf{Example:}
Find \(\frac{dy}{dx}\) if \(x^2 + xy + y^2 = 7\).
\begin{itemize}
    \item Differentiate both sides:
    \[
    \frac{d}{dx}(x^2) + \frac{d}{dx}(xy) + \frac{d}{dx}(y^2) = \frac{d}{dx}(7)
    \]
    \item Apply product rule and chain rule:
    \[
    2x + \left(\frac{d}{dx}(x \cdot y)\right) + 2y \frac{dy}{dx} = 0
    \]
    \[
    2x + \left(y + x \frac{dy}{dx}\right) + 2y \frac{dy}{dx} = 0
    \]
    \item Combine terms:
    \[
    2x + y + x \frac{dy}{dx} + 2y \frac{dy}{dx} = 0
    \]
    \item Solve for \(\frac{dy}{dx}\):
    \[
    \frac{dy}{dx} = -\frac{2x + y}{x + 2y}
    \]
\end{itemize}

\columnbreak

\section*{Integration}
\subsection*{Definition of Integration}

The definite integral of a function \( f(x) \) over the interval \([a, b]\) is defined as the limit of a Riemann sum:

\[
\int_{a}^{b} f(x) \, dx = \lim_{n \to \infty} \sum_{i=1}^{n} f(x_i^*) \Delta x
\]

where:
- \( \Delta x = \dfrac{b - a}{n} \) is the width of each subinterval,
- \( x_i^* \) is a sample point in the \( i \)-th subinterval \([x_{i-1}, x_i]\).

The integral represents the accumulation of the quantity \( f(x) \) over the interval \([a, b]\).

Alternatively, the indefinite integral (antiderivative) of a function \( f(x) \) is a function \( F(x) \) such that:

\[
\int f(x) \, dx = F(x) + C
\]

where:
- \( F'(x) = f(x) \),
- \( C \) is the constant of integration.

\subsection*{Power Rule}
\[
\int x^n \, dx = \frac{x^{n+1}}{n+1} + C \quad \text{(for } n \neq -1\text{)}
\]

\subsection*{Constant Rule}
\[
\int c \, dx = cx + C
\]
Example:
\[
\int 5x^2 \, dx = 5 \int x^2 \, dx = 5 \cdot \frac{x^3}{3} = \frac{5x^3}{3} + C
\]

\subsection*{Sum Rule}
\[
\int [f(x) + g(x)] \, dx = \int f(x) \, dx + \int g(x) \, dx
\]

\subsection*{Basic Integrals}
\[
\int e^x \, dx = e^x + C
\]
\[
\int a^x \, dx = \frac{a^x}{\ln a} + C \quad \text{(for } a > 0,\, a \neq 1\text{)}
\]
Example:
\[
\int 2^x \, dx = \frac{2^x}{\ln 2} + C
\]
\[
\int \frac{1}{x} \, dx = \ln |x| + C
\]
\[
\int \ln x \, dx = x \ln x - x + C
\]
\[
\int \log_a x \, dx = x \log_a x - \frac{x}{\ln a} + C
\]
Example:
\[
\int \log_2 x \, dx = x \log_2 x - \frac{x}{\ln 2} + C
\]

\subsection*{Integrals of Trigonometric Functions}
\[
\int \sin x \, dx = -\cos x + C
\]
\[
\int \cos x \, dx = \sin x + C
\]
\[
\int \tan x \, dx = -\ln |\cos x| + C
\]
\[
\int \cot x \, dx = \ln |\sin x| + C
\]
\[
\int \sec x \, dx = \ln |\sec x + \tan x| + C
\]
\[
\int \csc x \, dx = -\ln |\csc x + \cot x| + C
\]
\[
\int \sec^2 x \, dx = \tan x + C
\]
\[
\int \csc^2 x \, dx = -\cot x + C
\]
\[
\int \sec x \tan x \, dx = \sec x + C
\]
\[
\int \csc x \cot x \, dx = -\csc x + C
\]

\subsection*{Example Calculations}
1. \[
\int x^2 \, dx = \frac{x^3}{3} + C
\]
2. \[
\int 3^x \, dx = \frac{3^x}{\ln 3} + C
\]
3. \[
\int \ln x \, dx = x \ln x - x + C
\]
4. \[
\int \frac{1}{x \ln a} \, dx = \log_a x + C
\]
5. \[
\int \sec^2 x \, dx = \tan x + C
\]
6. \[
\int \tan x \, dx = -\ln |\cos x| + C
\]
\newpage

\subsection*{Substitution Method}

The substitution method is used to simplify an integral by making a substitution to reduce it to a standard form. It is particularly useful when the integrand contains a composite function.

\subsubsection*{Steps for Substitution}
\begin{enumerate}
    \item Identify a substitution: Let \( u = g(x) \), where \( g(x) \) is part of the integrand. Compute \( \frac{du}{dx} = g'(x) \) or equivalently \( du = g'(x) \, dx \).
    \item Rewrite the integral in terms of \( u \): Substitute \( g(x) \) with \( u \) and \( dx \) with \( du / g'(x) \).
    \item Perform the integration: Solve the integral in terms of \( u \).
    \item Back-substitute: Replace \( u \) with \( g(x) \) to express the answer in terms of the original variable \( x \).
\end{enumerate}

\subsubsection*{General Formula}
If \( \int (ax + b)^n \, dx \), then:
\[
\int (ax + b)^n \, dx = \frac{1}{a(n+1)} (ax + b)^{n+1} + C, \quad n \neq -1.
\]

\subsubsection*{Example 1}
Evaluate \( \int x e^{x^2} \, dx \).

\textbf{Solution:}
\begin{itemize}
    \item Let \( u = x^2 \), so \( \frac{du}{dx} = 2x \) or \( du = 2x \, dx \).
    \item Rewrite the integral: \[ \int x e^{x^2} \, dx = \int e^u \cdot \frac{du}{2}. \]
    \item Simplify and integrate: \[ \int e^u \cdot \frac{1}{2} \, du = \frac{1}{2} \int e^u \, du = \frac{1}{2} e^u + C. \]
    \item Back-substitute: \[ \frac{1}{2} e^u + C = \frac{1}{2} e^{x^2} + C. \]
\end{itemize}

\subsubsection*{Example 2}
Evaluate \( \int \frac{\ln x}{x} \, dx \).

\textbf{Solution:}
\begin{itemize}
    \item Let \( u = \ln x \), so \( \frac{du}{dx} = \frac{1}{x} \) or \( du = \frac{1}{x} \, dx \).
    \item Rewrite the integral: \[ \int \frac{\ln x}{x} \, dx = \int u \, du. \]
    \item Integrate: \[ \int u \, du = \frac{u^2}{2} + C. \]
    \item Back-substitute: \[ \frac{u^2}{2} + C = \frac{(\ln x)^2}{2} + C. \]
\end{itemize}

\subsubsection*{Example 3}
Evaluate \( \int \cos(3x) \, dx \).

\textbf{Solution:}
\begin{itemize}
    \item Let \( u = 3x \), so \( \frac{du}{dx} = 3 \) or \( du = 3 \, dx \), hence \( dx = \frac{du}{3} \).
    \item Rewrite the integral: \[ \int \cos(3x) \, dx = \int \cos(u) \cdot \frac{du}{3}. \]
    \item Simplify and integrate: \[ \int \cos(u) \cdot \frac{1}{3} \, du = \frac{1}{3} \int \cos(u) \, du = \frac{1}{3} \sin(u) + C. \]
    \item Back-substitute: \[ \frac{1}{3} \sin(u) + C = \frac{1}{3} \sin(3x) + C. \]
\end{itemize}
\end{multicols}

\end{document}
