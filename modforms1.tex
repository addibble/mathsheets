\documentclass[11pt]{article}
\usepackage{fontspec}
\usepackage{amsmath,amssymb,mathtools}
\usepackage[a4paper,margin=1in]{geometry}
\usepackage{enumitem}
\setlist[itemize]{noitemsep,topsep=0pt}
\usepackage{hyperref}

\title{Continued Fractions and Convergents}
\author{Prepared for a first exposure to modular forms}
\date{}

\begin{document}
\maketitle

\section*{Lesson 1: Continued Fractions and Quadratic Irrationals}

\paragraph{What we will build.}
We learn how quadratic irrationals like $\sqrt{23}$ have \emph{periodic} simple continued fraction (SCF) expansions. This is the gateway from concrete arithmetic to the geometry and dynamics that later underlie modular forms.

\paragraph{Historical vignette.}
Continued fractions grow out of the Euclidean algorithm (Euclid, \emph{Elements}, c.\ 300~BCE) for computing greatest common divisors. J.\ L.\ Lagrange (1768) proved that a real number has a periodic SCF if and only if it is a quadratic irrational. He was motivated by Diophantine approximation and the Pell equation. The term “Diophantine” comes from the ancient Greek mathematician Diophantus of Alexandria (3rd century CE), who studied equations involving only integers and introduced methods for solving them in his influential work \emph{Arithmetica}.
\smallskip
\noindent
The Pell equation is of the form
\[
x^2 - D y^2 = 1,
\]
where \( D \) is a fixed positive integer that is not a perfect square, and we seek integer pairs \( (x, y) \). Despite being named after John Pell, this equation was deeply studied in India centuries earlier by Brahmagupta (7th century) and Bhaskara II (12th century). Lagrange later showed that for every such \( D \), this equation has infinitely many integer solutions, and that these solutions can be found using the continued fraction expansion of \( \sqrt{D} \). The periodic nature of this expansion encodes the structure of those solutions.

\smallskip
\noindent
\textbf{Compact notation for continued fractions}

A continued fraction is an expression of the form:
\[
a_0 + \cfrac{1}{a_1 + \cfrac{1}{a_2 + \cfrac{1}{a_3 + \cdots}}}
\]
This is often written in compact notation as:
\[
[a_0;\, a_1, a_2, a_3, \dots]
\]
Here:
\begin{itemize}
  \item \( a_0 \) is the integer part of the number.
  \item \( a_1, a_2, \dots \) are called the \emph{partial quotients}.
  \item A bar over the entries indicates a repeating block: for example,
  \[
  [3;\, \overline{1,2,1,6}]
  \]
  means the sequence \( 1,2,1,6 \) repeats indefinitely.
\end{itemize}

This notation is useful because it compresses a complex nested expression into a short, readable list.

\paragraph{Periodicity in Non-Perfect-Squares}
For any non-perfect-square integer $D>0$ (where the square root is not an integer),
\[
\sqrt{D}=\bigl[a_0;\,\overline{a_1,a_2,\dots,a_\ell}\bigr],
\]
i.e., the simple continued fraction of $\sqrt{D}$ is eventually periodic, in fact purely periodic after $a_0=\lfloor\sqrt{D}\rfloor$.

\smallskip
\noindent
The expression $a_0 = \lfloor \sqrt{D} \rfloor$ uses the \emph{floor function} notation, where the symbols $\lfloor \cdot \rfloor$ denote the greatest integer less than or equal to the value inside. This is not to be confused with the square brackets $[\;]$ used in continued fraction notation.

\paragraph{Variable key.}
\begin{itemize}
  \item $D$ — A Diophantane number, a fixed, non-square positive integer whose square root we expand.
  \item $a_0$ — the integer part (floor) of $\sqrt{D}$, starting term of the SCF.
  \item $a_1,\dots,a_\ell$ — the repeating block of partial quotients.
  \item $\ell$ — length of the minimal period; its arithmetic reflects properties of $D$.
  \item Overline $\overline{\cdot}$ — indicates repetition of the enclosed block indefinitely.
\end{itemize}

\smallskip
\noindent
\textbf{Step-by-step algorithm to compute the continued fraction of \( \sqrt{D} \):}

Let \( \alpha_0 = \sqrt{D} \), and set
\[
a_0 = \lfloor \sqrt{D} \rfloor.
\]
Define recursively for \( k \geq 0 \):
\begin{align*}
m_{k+1} &= a_k d_k - m_k \\
d_{k+1} &= \frac{D - m_{k+1}^2}{d_k} \\
a_{k+1} &= \left\lfloor \frac{a_0 + m_{k+1}}{d_{k+1}} \right\rfloor \\
\alpha_{k+1} &= \frac{m_{k+1} + \sqrt{D}}{d_{k+1}}
\end{align*}
where the initial values are:
\[
m_0 = 0, \quad d_0 = 1, \quad \alpha_0 = \sqrt{D}, \quad a_0 = \lfloor \alpha_0 \rfloor.
\]
Each new complete quotient \( \alpha_k \) is of the form \( \frac{m_k + \sqrt{D}}{d_k} \), and the process repeats when the same pair \( (m_k, d_k) \) reappears.

\bigskip
\noindent
\textbf{Worked example: Continued fraction of \( \sqrt{17} \)}

We begin:
\[
\sqrt{17} \approx 4.123, \quad a_0 = \lfloor \sqrt{17} \rfloor = 4.
\]

\begin{itemize}
  \item Step 1:
  \[
  m_1 = 4, \quad d_1 = \frac{17 - 4^2}{1} = 1, \quad a_1 = \left\lfloor \frac{4 + 4}{1} \right\rfloor = 8
  \]
  Continued fraction so far: \( [4;\,8] \)

  \item Step 2:
  \[
  m_2 = 8 \cdot 1 - 4 = 4, \quad d_2 = \frac{17 - 4^2}{1} = 1, \quad a_2 = \left\lfloor \frac{4 + 4}{1} \right\rfloor = 8
  \]
  The pair \( (m_2, d_2) = (4, 1) \) has reappeared.

  \item So the continued fraction is:
  \[
  \sqrt{17} = [4;\, \overline{8}]
  \]
  with period length 1.
\end{itemize}
\smallskip
\noindent
In other words
\[
\sqrt{17} = [4;\, \overline{8}] = 4 + \cfrac{1}{8 + \cfrac{1}{8 + \cfrac{1}{8 + \cfrac{1}{\dots}}}}
\]

\paragraph{Workable examples.}
\begin{itemize}
  \item \textbf{Example A} Compute $\sqrt{14}$ until you find periodicity.
  \item \textbf{Example B} Compute $\sqrt{23}$ until you find periodicity.
\end{itemize}

\smallskip
\noindent
\textbf{What is a convergent?}

A \emph{convergent} is the rational approximation you get by truncating a continued fraction at a certain depth \( k \). If
\[
\sqrt{D} = [a_0;\, a_1, a_2, a_3, \dots],
\]
then the \( k \)-th convergent is:
\[
\frac{p_k}{q_k} = [a_0;\, a_1, a_2, \dots, a_k],
\]
i.e., the continued fraction stopped after the \( k \)-th term.

\smallskip
\textbf{Naive method:} You can compute \( \frac{p_k}{q_k} \) by evaluating the nested fraction directly from the beginning:
\[
[a_0;\, a_1, \dots, a_k] = a_0 + \cfrac{1}{a_1 + \cfrac{1}{\ddots + \cfrac{1}{a_k}}}
\]

\smallskip
\textbf{Shortcut (recurrence method):} Rather than recompute from scratch every time, we use a recursive formula:
\[
\begin{aligned}
p_{-1} &= 1,\quad p_0 = a_0, \quad p_k = a_k p_{k-1} + p_{k-2}, \\
q_{-1} &= 0,\quad q_0 = 1, \quad q_k = a_k q_{k-1} + q_{k-2}
\end{aligned}
\]

This way, we only need to know the two previous convergents.

\bigskip
\noindent
\textbf{Example: Compute the 6th convergent of \( \sqrt{14} \)}

We have:
\[
\sqrt{14} = [3;\, \overline{1,2,1,6}]
\]
So the continued fraction terms are:
\[
a_0 = 3,\quad a_1 = 1,\quad a_2 = 2,\quad a_3 = 1,\quad a_4 = 6,\quad a_5 = 1,\quad a_6 = 2,\dots
\]

We compute using the recurrence:

\[
\begin{aligned}
p_{-1} &= 1,\quad &q_{-1} &= 0 \\
p_0 &= 3,\quad &q_0 &= 1 \\
p_1 &= 1 \cdot 3 + 1 = 4,\quad &q_1 &= 1 \cdot 1 + 0 = 1 \\
p_2 &= 2 \cdot 4 + 3 = 11,\quad &q_2 &= 2 \cdot 1 + 1 = 3 \\
p_3 &= 1 \cdot 11 + 4 = 15,\quad &q_3 &= 1 \cdot 3 + 1 = 4 \\
p_4 &= 6 \cdot 15 + 11 = 101,\quad &q_4 &= 6 \cdot 4 + 3 = 27 \\
p_5 &= 1 \cdot 101 + 15 = 116,\quad &q_5 &= 1 \cdot 27 + 4 = 31 \\
p_6 &= 2 \cdot 116 + 101 = 333,\quad &q_6 &= 2 \cdot 31 + 27 = 89
\end{aligned}
\]

So the 6th convergent is:
\[
\frac{p_6}{q_6} = \frac{333}{89} \approx 3.7416 \quad \text{(very close to } \sqrt{14} \approx 3.7417\text{)}
\]

\smallskip
\noindent
\textbf{How continued fractions solve Pell’s equation}

Let \( \sqrt{D} = [a_0;\, \overline{a_1,\dots,a_\ell}] \), where \( \ell \) is the length of the minimal repeating period. Let \( \frac{p_k}{q_k} \) be the $k$-th convergent in the continued fraction expansion.

\begin{itemize}
  \item If the period length \( \ell \) is even, then \( (x, y) = (p_{\ell - 1}, q_{\ell - 1}) \) is the minimal solution to
  \[
  x^2 - D y^2 = 1.
  \]
  \item If the period length \( \ell \) is odd, then \( (p_{\ell - 1}, q_{\ell - 1}) \) solves
  \[
  x^2 - D y^2 = -1,
  \]
  and squaring this gives the minimal solution to
  \[
  x^2 - D y^2 = 1
  \]
  at \( (x, y) = (p_{2\ell - 1}, q_{2\ell - 1}) \).
\end{itemize}

\textbf{Example:} \( \sqrt{17} = [4;\, \overline{8}] \) has period length \( \ell = 1 \) (odd), so we compute two full periods. The first convergent is
\[
\frac{4}{1}, \quad \text{and } 4^2 - 17 \cdot 1^2 = -1.
\]
Squaring gives:
\[
(4 + \sqrt{17})^2 = 33 + 8\sqrt{17},
\]
so \( (x, y) = (33, 8) \) solves the Pell equation \( x^2 - 17 y^2 = 1 \).

\section*{Summary and Next Steps}

We have seen that every non-square integer \(D\) yields a periodic simple continued fraction
\[
\sqrt{D}=[a_0;\overline{a_1,\dots,a_\ell}],
\]
whose convergents \(\tfrac{p_k}{q_k}\) are not only excellent approximations to \(\sqrt{D}\) but also encode the minimal solutions to
\[
x^2 - D y^2 = 1.
\]
In particular, truncating at one full period (or two, when the period length is odd) produces the fundamental unit in the real quadratic field \(\mathbb{Q}(\sqrt{D})\).

Why is this cool?  Because beneath this arithmetic story lies the action of the modular group \(\mathrm{SL}_2(\mathbb{Z})\) on the upper half-plane:
\[
\gamma=\begin{pmatrix}a&b\\c&d\end{pmatrix}:\; \tau\mapsto\frac{a\tau+b}{c\tau+d}.
\]
Continued fractions arise from the simple generators of \(\mathrm{SL}_2(\mathbb{Z})\), and the closed orbits you’ve just computed correspond to closed geodesics on the modular surface.  

In our next lesson, we will introduce \textbf{modular forms}—functions on the upper half-plane that transform in a precise way under \(\mathrm{SL}_2(\mathbb{Z})\) and admit Fourier (or \(q\)-) expansions.  You’ll see how the same arithmetic that produced Pell’s solutions now reappears in the coefficients of these spectacular analytic objects.

\end{document}
