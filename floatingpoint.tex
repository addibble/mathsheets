\documentclass[12pt]{article}
\usepackage[utf8]{inputenc}
\usepackage[T1]{fontenc}
\usepackage{lmodern}
\usepackage{amsmath}
\usepackage{geometry}
\geometry{margin=1in}
\usepackage{hyperref}

\title{Understanding IEEE 754 Floating Point}
\author{}
\date{}

\begin{document}

\maketitle

\section*{1. Introduction}
Floating-point numbers let computers represent very large, very small, and fractional values using a fixed number of bits. IEEE 754 is the standard that defines how these bits are divided into \emph{sign}, \emph{exponent}, and \emph{mantissa}. In this lesson you'll learn:
\begin{itemize}
  \item How the IEEE 754 format is structured and why the exponent uses a \emph{bias}.
  \item Step-by-step conversion from decimal to IEEE 754 binary (16-bit) and back.
  \item Two detailed examples worked through each substep.
  \item Eight practice problems to reinforce your understanding.
\end{itemize}

\section*{2. IEEE 754 Format Overview}
An IEEE~754 floating-point number is laid out as:
\[
\underbrace{\text{Sign}}_{1\text{ bit}}
\;|\;
\underbrace{\text{Exponent}}_{E\text{ bits}}
\;|\;
\underbrace{\text{Mantissa}}_{M\text{ bits}}
\]
\begin{itemize}
  \item \textbf{Sign bit (S)}: 0 for positive, 1 for negative.
  \item \textbf{Exponent}: stored with a \emph{bias} so that both positive and negative exponents become nonnegative integers.
  \item \textbf{Mantissa} (also called \emph{fraction}): the significant digits, stored without bias, with an \emph{implicit leading 1} in normalized numbers.
\end{itemize}

For common precisions:
\begin{center}
\begin{tabular}{l|c|c|c}
Precision & Total bits & Exponent bits ($E$) & Mantissa bits ($M$) \\
\hline
Half (16-bit)   & 16 & 5 & 10 \\
Single (32-bit) & 32 & 8 & 23 \\
Double (64-bit) & 64 & 11 & 52 \\
\end{tabular}
\end{center}

\subsection*{2.1 Why a Bias?}
Exponents can be positive or negative. Instead of using a separate sign bit for the exponent, IEEE~754 adds a \textbf{bias} to make the stored exponent always nonnegative:
\[
e_{\text{stored}} = e_{\text{actual}} + (2^{E-1}-1)
\]
For $E=5$, the bias is $2^{4}-1=15$. For example:
\[
\begin{array}{c|c}
e_{\text{actual}} & e_{\text{stored}} \\
\hline
-15 & 0 \\
-14 & 1 \\
\vdots & \vdots \\
0   & 15 \\
1   & 16 \\
\vdots & \vdots \\
15  & 30 \\
16  & 31 \\
\end{array}
\]
Storing exponents as unsigned integers simplifies hardware comparison and sorting.

\section*{3. Converting Decimal to IEEE~754 (16-bit)}
We will convert $6.75_{10}$ step by step.

\subsection*{Step 1: Sign Bit}
Since $6.75$ is positive, $S=0$.

\subsection*{Step 2: Convert to Binary}
Split into integer part (6) and fractional part ($0.75$).

\subsubsection*{2.1 Integer Part (repeated division by 2)}
\begin{align*}
6 \div 2 &= 3 \quad\text{remainder }0 \\
3 \div 2 &= 1 \quad\text{remainder }1 \\
1 \div 2 &= 0 \quad\text{remainder }1
\end{align*}
Reading the remainders \textbf{bottom to top} gives $6_{10} = 110_2$.

\subsubsection*{2.2 Fractional Part (repeated multiplication by 2)}
\begin{align*}
0.75 \times 2 &= 1.50 \quad\text{write 1, remainder }0.50 \\
0.50 \times 2 &= 1.00 \quad\text{write 1, remainder }0.00
\end{align*}
So $0.75_{10} = .11_2$. Combine: $6.75_{10} = 110.11_2$.

\subsection*{Step 3: Normalize}
Move the binary point so the form is $1.xxxxx \times 2^e$:
\[
110.11_2 = 1.1011_2 \times 2^{2}
\]
We moved the point left $2$ places, so $e_{\text{actual}} = 2$.

\subsection*{Step 4: Biased Exponent}
Bias is $15$, so
\[
e_{\text{stored}} = e_{\text{actual}} + 15 = 2 + 15 = 17 = 10001_2
\]

\subsection*{Step 5: Mantissa}
Drop the leading 1, take the next 10 bits of $1.1011$ (pad with zeros if needed): \texttt{1011000000}.

\subsection*{Step 6: Assemble}
So the IEEE 754 16-bit representation is:
\[
0\ |\ 10001\ |\ 1011000000
\]

\subsection*{Example: Convert $-2.625$}

\begin{enumerate}
  \item $S=1$ (negative).
  \item $2.625 = 10.101_2$:
    \begin{itemize}
      \item $2_{10} = 10_2$
      \item $0.625 \times 2 = 1.25$ (write 1, remainder 0.25)\\
      $0.25 \times 2 = 0.5$ (write 0, remainder 0.5)\\
      $0.5 \times 2 = 1.0$ (write 1, remainder 0.0)\\
      So $0.625_{10} = .101_2$.
    \end{itemize}
  \item Normalize: $10.101_2 = 1.0101_2 \times 2^{1}$
  \item Biased exponent: $e_{\text{stored}} = 1 + 15 = 16 = 10000_2$
  \item Mantissa: drop the leading 1, next 10 bits: \texttt{0101000000}
  \item Final: $1\ |\ 10000\ |\ 0101000000$
\end{enumerate}

\section*{4. Converting IEEE~754 Back to Decimal}
Given $(S, e_{\text{stored}}, \text{mantissa bits})$:
\begin{align*}
\text{Sign} &= (-1)^S \\
e &= e_{\text{stored}} - \text{Bias} \\
M &= 1 + \sum_{i=1}^{10} \text{bit}_i \cdot 2^{-i} \\
\text{Value} &= \text{Sign} \times M \times 2^e
\end{align*}

\subsection*{Example: $0\ |\ 10001\ |\ 1011000000$}
\begin{enumerate}
  \item $S=0$, so positive.
  \item $e_{\text{stored}} = 17$, so $e = 17 - 15 = 2$.
  \item Mantissa $= 1 + 2^{-1} + 0 \cdot 2^{-2} + 1 \cdot 2^{-3} + 1 \cdot 2^{-4} = 1 + 0.5 + 0 + 0.125 + 0.0625 = 1.6875$
  \item Value $= 1.6875 \times 2^2 = 6.75$
\end{enumerate}

\subsection*{Example: $1\ |\ 10000\ |\ 0101000000$}
\begin{enumerate}
  \item $S=1$, so negative.
  \item $e_{\text{stored}} = 16$, so $e = 16 - 15 = 1$.
  \item Mantissa $= 1 + 0 \cdot 2^{-1} + 1 \cdot 2^{-2} + 0 \cdot 2^{-3} + 1 \cdot 2^{-4} = 1 + 0 + 0.25 + 0 + 0.0625 = 1.3125$
  \item Value $= -1.3125 \times 2^1 = -2.625$
\end{enumerate}

\section*{5. Practice Problems}
Convert with all steps:
\begin{enumerate}
  \item Convert $3.125$ to 16-bit IEEE 754 and back.
  \item Convert $-0.15625$ to 16-bit IEEE 754 and back.
  \item Convert $12.5$ to 16-bit IEEE 754.
  \item Given $0\ |\ 01110\ |\ 0010000000$, convert to decimal.
  \item Convert $-7.75$ to 16-bit IEEE 754.
  \item Convert $0.1$ to 16-bit IEEE 754 (approximate).
  \item Convert $15.875$ to 16-bit IEEE 754.
  \item Explain why $1/10$ is infinite in binary.
\end{enumerate}

\end{document}
