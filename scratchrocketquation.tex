\documentclass[12pt]{article}
\usepackage{amsmath}
\usepackage{graphicx}
\usepackage{geometry}
\geometry{margin=1in}
\title{Scratch Rocket Simulator Manual with Gravity Turn and Altitude-Dependent Air Density}
\author{Kerbal Space Math Adventure}
\date{}

\begin{document}

\maketitle

\section{Introduction}
This manual explains a Scratch-based rocket simulator that uses Euler's method to approximate the rocket's trajectory. The simulator accounts for:
\begin{itemize}
    \item Engine thrust and mass loss (fuel burn)
    \item Gravitational acceleration
    \item Air resistance (drag) with an altitude-dependent air density model
    \item A gravity turn maneuver for an optimized launch trajectory
\end{itemize}

\section{Key Variables and Their Descriptions}
Below is a list of all variables used in the simulation, along with their meanings and units:

\begin{itemize}
    \item \textbf{T (thrust):} Engine thrust, measured in newtons (N). In the simulator, this is represented by the variable \texttt{thrust}.
    \item \textbf{m (mass):} Current mass of the rocket (rocket + remaining fuel) in kilograms (kg). Represented by \texttt{mass}. This decreases as fuel is burned.
    \item \textbf{fuel:} Remaining fuel mass in kilograms (kg). Represented by \texttt{fuel}.
    \item \textbf{$\dot{m}$ (burnRate):} Fuel consumption rate in kg/s. Represented by \texttt{burnRate}.
    \item \textbf{g (gravity):} Gravitational acceleration, set to 9.81 m/s\(^2\). Represented by \texttt{gravity}.
    \item \textbf{$\theta$ (angle):} The current pitch (launch) angle in degrees. Initially 90° for a vertical launch, then adjusted during the gravity turn. Represented by \texttt{angle}.
    \item \textbf{rad:} The angle in radians (converted from \texttt{angle} using 
    \[
      \text{rad} = \theta \times \frac{\pi}{180}
    \]
    ).
    \item \textbf{$v_x,\, v_y$ (vx, vy):} Horizontal and vertical components of velocity in m/s.
    \item \textbf{$x,\, y$ (x, y):} Horizontal and vertical positions in meters (m).
    \item \textbf{$\Delta t$ (dt):} Time step for Euler integration in seconds (s). Represented by \texttt{dt}.
    \item \textbf{$\rho$ (air density):} Air density in kg/m\(^3\). At sea level, \(\rho_0 = 1.225\) kg/m\(^3\). In our model, it decreases with altitude according to 
    \[
      \rho(y) = \rho_0 \, e^{-y/h},
    \]
    where \(h\) is the scale height (taken as 5000 m). In the simulator, the current air density is stored in \texttt{currentAirDensity}.
    \item \textbf{$C_d$ (dragCoefficient):} Drag coefficient (dimensionless). Represented by \texttt{dragCoefficient}.
    \item \textbf{A (area):} Frontal (cross-sectional) area of the rocket in m\(^2\). Represented by \texttt{area}.
    \item \textbf{$F_{\text{drag}}$:} Drag force in newtons (N), calculated as:
    \[
      F_{\text{drag}} = \frac{1}{2} \, \rho(y) \, v^2 \, C_d \, A.
    \]
    \item \textbf{$a_{\text{drag}}$:} Drag acceleration in m/s\(^2\), found by dividing the drag force by the mass:
    \[
      a_{\text{drag}} = \frac{F_{\text{drag}}}{m}.
    \]
    Its components (acting opposite to the direction of motion) are:
    \[
      a_{dx} = -a_{\text{drag}} \cdot \frac{v_x}{v}, \quad a_{dy} = -a_{\text{drag}} \cdot \frac{v_y}{v},
    \]
    where 
    \[
      v = \sqrt{v_x^2 + v_y^2}.
    \]
    \item \textbf{Gravity Turn Parameters:}
    \begin{itemize}
        \item \texttt{turnStart}: Altitude (in m) at which the gravity turn begins.
        \item \texttt{turnEnd}: Altitude (in m) at which the gravity turn is completed.
        \item \texttt{finalAngle}: The final pitch angle (in degrees) after the gravity turn, denoted as \(\theta_{\text{final}}\).
    \end{itemize}
\end{itemize}

\section{The Core Formulas}
\subsection*{Thrust Acceleration}
The acceleration produced by the engine's thrust is given by:
\[
a_{x,\text{thrust}} = \frac{T \cos(\text{rad})}{m}, \quad 
a_{y,\text{thrust}} = \frac{T \sin(\text{rad})}{m} - g.
\]
Here, \(T\) is thrust, \(\theta\) is the pitch angle, and \(\text{rad}\) is the angle in radians.

\subsection*{Drag Force and Acceleration}
The drag force is:
\[
F_{\text{drag}} = \frac{1}{2} \, \rho(y) \, v^2 \, C_d \, A,
\]
and the corresponding acceleration is:
\[
a_{\text{drag}} = \frac{F_{\text{drag}}}{m}.
\]
Its components, acting opposite to the direction of velocity, are:
\[
a_{dx} = -a_{\text{drag}} \frac{v_x}{v}, \quad 
a_{dy} = -a_{\text{drag}} \frac{v_y}{v}.
\]

\subsection*{Total Acceleration}
The net acceleration is the sum of thrust acceleration and drag acceleration:
\[
a_x = a_{x,\text{thrust}} + a_{dx}, \quad 
a_y = a_{y,\text{thrust}} + a_{dy}.
\]

\subsection*{Euler Integration Update Equations}
The state of the rocket is updated every time step \(\Delta t\) as follows:
\[
v_x(t+\Delta t) = v_x(t) + a_x \, \Delta t, \quad 
v_y(t+\Delta t) = v_y(t) + a_y \, \Delta t,
\]
\[
x(t+\Delta t) = x(t) + v_x(t) \, \Delta t, \quad 
y(t+\Delta t) = y(t) + v_y(t) \, \Delta t,
\]
\[
m(t+\Delta t) = m(t) - (\text{burnRate} \cdot \Delta t).
\]

\subsection*{Gravity Turn (Dynamic Pitch Update)}
When the rocket reaches an altitude above \texttt{turnStart}, the pitch angle is updated by linear interpolation:
\[
\theta = 90^\circ - \left(90^\circ - \theta_{\text{final}}\right) \cdot \frac{y - y_{\text{start}}}{y_{\text{end}} - y_{\text{start}}},
\]
and clamped so that \(\theta \geq \theta_{\text{final}}\).

\section{Scratch Implementation Pseudocode}
Below is the pseudocode for the complete Scratch simulator,
now including altitude-dependent air density, drag, and a gravity turn:
\begin{verbatim}
when green flag clicked
    // Initialize variables:
    set [mass v] to 1000        // in kg (rocket + fuel)
    set [fuel v] to 400         // in kg
    set [thrust v] to 15000     // in N (T)
    set [burnRate v] to 10      // in kg/s (fuel consumption rate, m dot)
    set [gravity v] to 9.81     // in m/s^2 (g)
    set [angle v] to 90         // initial pitch angle in degrees (theta)
    set [vx v] to 0             // horizontal velocity in m/s
    set [vy v] to 0             // vertical velocity in m/s
    set [x v] to 0              // horizontal position in m
    set [y v] to 0              // vertical position in m
    set [dt v] to 0.2           // time step in seconds (deltat)
    
    // Drag and atmosphere variables:
    set [dragCoefficient v] to 0.75  // (C_d), dimensionless
    set [area v] to 1.0              // frontal area in m^2 (A)
    // Base air density at sea level: 1.225 kg/m^3
    
    // Gravity turn parameters:
    set [turnStart v] to 500    // altitude in m at which to
    					 // start the turn (y_start)
    set [turnEnd v] to 5000     // altitude in m at which the turn
    					   // is complete (y_end)
    set [finalAngle v] to 45    // final pitch angle in degrees (theta_final)

    repeat until <(fuel) <= 0 or (y) < 0>
        // Gravity turn: update angle based on altitude:
        if <(y) > (turnStart)> then
            set [angle v] to (90 - ((90 - finalAngle) * ((y - turnStart)
            				/ (turnEnd - turnStart))))
            if <(angle) < (finalAngle)> then
                set [angle v] to (finalAngle)
            end
        end
        // Convert angle to radians:
        set [rad v] to ((angle) * 3.14159 / 180)
        
        // Update current air density: rho(y) = 1.225 * e^( - y / 5000 )
        set [currentAirDensity v] to (1.225 * (e^( - (y / 5000) )))
        
        // Calculate thrust acceleration components:
        set [ax_thrust v] to ((thrust * cos(rad)) / mass)
        set [ay_thrust v] to (((thrust * sin(rad)) / mass) - gravity)
        
        // Compute current speed: v = sqrt((vx^2 + vy^2))
        set [v_total v] to (sqrt((vx)^2 + (vy)^2))
        
        // Calculate drag if v_total > 0:
        if <(v_total) > (0)> then
            set [F_drag v] to (0.5 * currentAirDensity * ((v_total)^2)
            				* dragCoefficient * area)
            set [a_drag v] to (F_drag / mass)
            set [ax_drag v] to ((-1) * a_drag * (vx / v_total))
            set [ay_drag v] to ((-1) * a_drag * (vy / v_total))
        else
            set [ax_drag v] to 0
            set [ay_drag v] to 0
        end
        
        // Total acceleration:
        set [ax v] to (ax_thrust + ax_drag)
        set [ay v] to (ay_thrust + ay_drag)
        
        // Update velocities using Euler's method:
        set [vx v] to (vx + ax * dt)
        set [vy v] to (vy + ay * dt)
        
        // Update positions:
        set [x v] to (x + vx * dt)
        set [y v] to (y + vy * dt)
        
        // Update fuel and mass:
        change [fuel v] by ((-burnRate) * dt)
        change [mass v] by ((-burnRate) * dt)
        
        // Move rocket sprite (optional):
        go to x: (x) y: (y)
        wait (dt) seconds
    end
\end{verbatim}

\section{Conclusion}
This manual details a comprehensive Scratch-based rocket simulator that:
\begin{itemize}
    \item Uses Euler's method for numerical integration.
    \item Accounts for thrust, fuel burn, gravity, drag with an altitude-dependent atmosphere, and a dynamic gravity turn.
\end{itemize}
Every variable and formula is clearly labeled and described, allowing you to experiment with and understand the physics behind rocket flight. Enjoy exploring and optimizing your simulated launch trajectories!

\end{document}
