\documentclass[11pt]{article}
\usepackage{fontspec}
\usepackage{amsmath,amssymb,mathtools}
\usepackage[a4paper,margin=1in]{geometry}
\usepackage{enumitem}
\setlist[itemize]{noitemsep,topsep=0pt}
\usepackage{hyperref}

\title{Möbius Transformations and the Modular Action}
\author{Prepared for an introduction to modular symmetries}
\date{}

\begin{document}
\maketitle

\section*{Part 1: Möbius Transformations on the Real Line}

\paragraph{Historical vignette.}
Möbius transformations date back to August Ferdinand Möbius (1827), who studied the maps of the form
\[
x \;\longmapsto\; \frac{a\,x + b}{c\,x + d}
\]
and observed their importance in projective geometry.  They generalize familiar moves like translations, inversions, and reflections.

\paragraph{Notation and variables.}
\begin{itemize}
  \item \(x\): a point on the real line \(\mathbb{R}\).
  \item \(a,b,c,d\in\mathbb{R}\): real parameters, with \(ad - bc\neq 0\) to ensure invertibility.
  \item We write the transformation as
    \[
      M(x) = \frac{a\,x + b}{c\,x + d}\,.
    \]
\end{itemize}

\paragraph{Step‐by‐step algorithm on \(\mathbb{R}\).}
Given \(x\in\mathbb{R}\) and parameters \((a,b,c,d)\):
\begin{enumerate}
  \item Compute the numerator \(N = a\,x + b\).
  \item Compute the denominator \(D = c\,x + d\).
  \item Form the quotient \(M(x) = N/D\).
\end{enumerate}

\paragraph{Matrix viewpoint.}
Observe that
\[
\begin{pmatrix}a & b\\ c & d\end{pmatrix}
\quad\longleftrightarrow\quad
x \longmapsto \frac{a\,x + b}{c\,x + d}.
\]
The determinant \(\det = ad - bc\neq0\) ensures this is an invertible map.

\paragraph{Worked example.}
Let
\[
M(x) = \frac{2x - 3}{x + 1}, 
\quad\text{matrix }
\begin{pmatrix}2 & -3\\[6pt]1 & 1\end{pmatrix}.
\]
Compute \(M(4)\):
\[
N = 2\cdot4 - 3 = 5,\quad D = 1\cdot4 + 1 = 5,
\quad M(4) = \frac{5}{5} = 1.
\]
Verify invertibility by checking \(\det = 2\cdot1 - (-3)\cdot1 = 5\neq0\).

\paragraph{Practice problems.}
\begin{enumerate}
  \item Compute \(M(x)=\frac{3x+2}{2x-1}\) at \(x=1\) and \(x=2\).  
  \item Verify invertibility of \(N(x)=\tfrac{4x-5}{x+2}\) by computing its determinant and then finding \(N^{-1}(y)\).
\end{enumerate}

\newpage
\section*{Part 2: \(\mathrm{SL}_2(\mathbb{Z})\) Action on the Upper Half‐Plane}

\paragraph{Historical vignette.}
The group \(\mathrm{SL}_2(\mathbb{Z})\) — 2×2 integer matrices of determinant 1 — was first studied in earnest by Felix Klein and Henri Poincaré in the late 19th century in connection with tessellations of the hyperbolic plane.

\paragraph{Definitions and notation.}
\begin{itemize}
  \item \(\displaystyle \mathrm{SL}_2(\mathbb{Z})
    = \Bigl\{\begin{pmatrix}a&b\\c&d\end{pmatrix}\!
      \Bigm|\;a,b,c,d\in\mathbb{Z},\;ad-bc=1\Bigr\}.\)
  \item Upper half‐plane:
    \(\displaystyle \mathbb{H} = \{\,\tau\in\mathbb{C}\mid \Im(\tau)>0\}.\)
  \item Action:
    \[
      \gamma = \begin{pmatrix}a & b\\ c & d\end{pmatrix}
      \;:\;\tau \;\longmapsto\;
      \gamma\cdot\tau \;=\;
      \frac{a\,\tau + b}{c\,\tau + d}.
    \]
\end{itemize}

\paragraph{Key generators.}
\[
T = \begin{pmatrix}1&1\\0&1\end{pmatrix},\quad
S = \begin{pmatrix}0&-1\\1&0\end{pmatrix},
\]
with relations \(S^2 = (ST)^3 = -I\).

\paragraph{Worked example.}
Let \(\tau = i\).  Compute
\[
T\cdot i = i + 1,\quad
S\cdot i = -\tfrac1{i} = i.
\]
Now \(\displaystyle ST\cdot i = S(i+1)= -\tfrac1{i+1} = \frac{i-1}{2}\).  Check \(\Im((i-1)/2)>0\).

\paragraph{Practice problems.}
\begin{enumerate}
  \item Compute \(\displaystyle T^3\cdot i\) and \(S\,T^2\,S\cdot i\).
  \item For \(\gamma=\begin{pmatrix}2&1\\1&1\end{pmatrix}\), evaluate \(\gamma\cdot(2i)\) and verify the result lies in \(\mathbb{H}\).
\end{enumerate}

\newpage
\section*{Part 3: Continued Fractions Encoding \(\mathrm{SL}_2(\mathbb{Z})\) Action}

\paragraph{Bridge remark.}
Each simple continued fraction step
\([a_0;\,a_1,\dots,a_k]\)
corresponds to a product of the generators \(T\) and \(S\):
\[
[a_0] \;\longleftrightarrow\; T^{a_0}, 
\quad
[a_0;a_1] \;\longleftrightarrow\; T^{a_0}S T^{a_1}, 
\quad\text{etc.}
\]

\paragraph{Notation.}
\[
[a_0;\,a_1,\ldots,a_k]
\;\longmapsto\;
\underbrace{T^{a_0}S T^{a_1}S\cdots S T^{a_k}}_{\in\mathrm{SL}_2(\mathbb{Z})}.
\]

\paragraph{Worked example.}
\[
\sqrt{2}=[1;\overline{2}]
\;\longmapsto\;
T^1\,S\,T^2\,S
=\begin{pmatrix}1&1\\0&1\end{pmatrix}
\begin{pmatrix}0&-1\\1&0\end{pmatrix}
\begin{pmatrix}1&2\\0&1\end{pmatrix}
\begin{pmatrix}0&-1\\1&0\end{pmatrix}.
\]
Multiplying out gives a matrix whose fixed point is \(\frac{1+\sqrt2}2\).

\paragraph{Practice problems.}
\begin{enumerate}
  \item Write the continued fraction \([2;1,3]\) as a product of \(T\) and \(S\).
  \item For \(\alpha = [3;\overline{1,4}]\), form the corresponding matrix and verify it fixes \(\alpha\).
\end{enumerate}

\end{document}
