\documentclass{article}
\usepackage[utf8]{inputenc} % Handle UTF-8 encoding
\usepackage{amsmath, amssymb, tikz, geometry, multicol}
\usetikzlibrary{calc}
\geometry{margin=0.15in}
\tolerance=1000

\begin{document}

% Start of the document with two columns
\begin{multicols}{2}

% Left Column: Summations and Series
\section*{Summations and Series}

\subsection*{Summation Notation}
The summation symbol \(\Sigma\) represents the sum of a sequence:
\[
\sum_{i=m}^n a_i = a_m + a_{m+1} + \cdots + a_n
\]
\textbf{Example:} Compute \( \sum_{i=1}^4 i \):
\[
\sum_{i=1}^4 i = 1 + 2 + 3 + 4 = 10
\]

\section*{Arithmetic Series}
An arithmetic series is the sum of terms in an arithmetic sequence, with a constant difference \(d\).

\textbf{General Form:}
\[
S_n = \frac{n}{2}(a + l) = \frac{n}{2}(2a + (n-1)d)
\]
where:
\begin{itemize}
    \item \(a\): First term
    \item \(l\): Last term
    \item \(d\): Common difference
    \item \(n\): Number of terms
\end{itemize}

\textbf{Example:} Sum of first 10 terms of \(2, 5, 8, \dots\):
\begin{align*}
    a &= 2, d = 3, n = 10 \\
    S_n &= \frac{10}{2}(4 + 27) = 155
\end{align*}

\section*{Geometric Series}
A geometric series is the sum of terms in a geometric sequence, with a constant ratio \(r\).

\textbf{Finite Sum:}
\[
S_n = a \frac{1-r^n}{1-r}, \quad r \neq 1
\]

\textbf{Infinite Sum:} If \(|r| < 1\):
\[
S = \frac{a}{1-r}
\]

\textbf{General term:}
\[ a_n = a r^{n-1} \]

\textbf{Example:} Sum of first 5 terms of \(3, 6, 12, \dots\):
\begin{align*}
    a &= 3, r = 2, n = 5 \\
    S_n &= 3\frac{1 - 32}{1 - 2} = 93
\end{align*}

\section*{Partial Sums of Infinite Series}
The partial sum \(S_N\) is the sum of the first \(N\) terms:
\[
S_N = \sum_{n=1}^{N} a_n
\]
For geometric series:
\[
S_N = a \frac{1 - r^N}{1 - r}, |r|<1
\]

As \(N \to \infty\), the sum approaches:
\[
S = \frac{a}{1-r}
\]

\textbf{Example:} Sum of first 4 terms of \(1, \frac{1}{2}, \frac{1}{4}, \dots\):
\begin{align*}
    a &= 1, r = \frac{1}{2}, N = 4 \\
    S_N &= \frac{1-(\frac{1}{2})^4}{1-\frac{1}{2}} = \frac{15}{8}
\end{align*}

% Right Column: Other Formulas
\section*{Conversions and Other Formulas}

\subsection*{Converting Recursive to Explicit Form}
\textbf{Arithmetic:}
\[ a_n = a + (n-1)d \]

\textbf{Geometric:}
\[ a_n = a r^{n-1} \]

\section*{Product Series Notation}
The product notation \(\Pi\) represents the product of terms:
\[
\prod_{i=m}^{n} a_i = a_m \times a_{m+1} \times \dots \times a_n
\]

\textbf{Example:} Compute \(\prod_{i=1}^{4} i\):
\[
\prod_{i=1}^{4} i = 1 \times 2 \times 3 \times 4 = 24
\]

\subsection*{Infinite Product}
An infinite product converges if the limit exists:
\[
\prod_{i=1}^{\infty} a_i = \lim_{n\to\infty}\prod_{i=1}^{n}a_i
\]

\textbf{Example:} Evaluate the infinite product \(\prod_{n=1}^{\infty} \frac{1}{2^n}\):
\[
\prod_{n=1}^{\infty}\frac{1}{2} = \lim_{n\to\infty} \frac{1}{2^n} = 0
\]

\end{multicols}

\end{document}
