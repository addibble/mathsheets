\documentclass[12pt]{article}
\usepackage{amsmath,amssymb,bm}
\usepackage{physics}    % \ket, \bra, \braket, \norm, etc.
\usepackage{microtype}
\usepackage{geometry}
\geometry{margin=1in}

\title{\bfseries Single-Qubit Introduction}
\author{}
\date{}

\begin{document}
\maketitle
\vspace{-2em}

%%%%%%%%%%%%%%%%%%%%%%%%%%%%%%%%%%%%%%%%%%%%%%%%%%%%
\section*{1 \quad What Is a Qubit?}
A qubit is the \emph{quantum analogue} of a classical bit.  Mathematically
it is a \textbf{unit vector} in a two–dimensional complex vector space.
Dirac’s \textit{bra–ket} notation writes such vectors as
\[
   \ket{\psi} \;=\; \alpha\ket{0}+\beta\ket{1},
   \qquad
   \alpha,\beta\in\mathbb{C},\;
   \abs{\alpha}^{2}+\abs{\beta}^{2}=1.
\]
The Greek letter $\psi$ (psi) labels the state; the vertical bar and
angle bracket form a \emph{ket}.

%%%%%%%%%%%%%%%%%%%%%%%%%%%%%%%%%%%%%%%%%%%%%%%%%%%%
\section*{2 \quad Why Those Particular Column Vectors?}

\begin{itemize}
  \item $\ket{0}\!=\!\begin{bmatrix}1\\0\end{bmatrix}$ and
        $\ket{1}\!=\!\begin{bmatrix}0\\1\end{bmatrix}$ are chosen as the
        \emph{computational basis}.  They are
        \textbf{orthonormal}—see \S\ref{subsec:orthonormal-definition}.
  \item Any qubit state can be written as a linear combination
        $\alpha\ket{0}+\beta\ket{1}$ because $\{\ket{0},\ket{1}\}$ is a
        complete basis for $\mathbb{C}^{2}$.
\end{itemize}

\subsection*{(What does \textit{orthonormal} mean?)}%
\label{subsec:orthonormal-definition}

Let $\ket{u}$ and \(\ket{v}\) be column vectors in a complex vector
space.

\paragraph{1. Conjugate transpose (\texorpdfstring{$\dagger$}{†}).}
Given a column vector
$
  \ket{u}=\begin{bmatrix}u_{1}\\u_{2}\\\vdots\\u_{n}\end{bmatrix},
$
its \textbf{conjugate transpose} (also called the
\emph{Hermitian adjoint}) is
\[
   \ket{u}^{\dagger}
   \;=\;
   \begin{bmatrix}
     u_{1}^{*} & u_{2}^{*} & \dots & u_{n}^{*}
   \end{bmatrix},
\]
where \(\,*\) denotes complex conjugation.  In words:
\begin{enumerate}
  \item \emph{Transpose}: turn the column into a row.
  \item \emph{Conjugate}: replace each entry \(a+ib\) with \(a-ib\).
\end{enumerate}
For a matrix \(A\), \(A^{\dagger}=(A^{T})^{*}\)---transpose first, then
conjugate every element.

\paragraph{2. Inner product.}
The \emph{Dirac inner product} of \(\ket{u}\) and \(\ket{v}\) is
\[
   \braket{u}{v}
   \;=\;
   \ket{u}^{\dagger}\ket{v}
   \;=\;
   \sum_{i=1}^{n}u_{i}^{*}\,v_{i}.
\]

\paragraph{3. Orthogonality.}
Vectors are \textbf{orthogonal} when their inner product vanishes:
\[
   \braket{u}{v}=0.
\]

\paragraph{4. Normalisation.}
A vector is \textbf{normalised} when its inner product with itself
equals 1:
\[
   \braket{u}{u}=1.
\]
Its \textbf{norm} (length) is \(\norm{\ket{u}}=\sqrt{\braket{u}{u}}\).

\paragraph{5. Orthonormality.}
A set of vectors is \emph{orthonormal} if every pair is orthogonal and
every vector is normalised:
\[
   \braket{u}{v}=0,\qquad
   \braket{u}{u}=1,\;
   \braket{v}{v}=1,\;\dots
\]

\paragraph{Example (computational basis).}
\[
   \braket{0}{1}
   =\begin{bmatrix}1&0\end{bmatrix}
     \begin{bmatrix}0\\1\end{bmatrix}
   =0,
   \qquad
   \braket{0}{0}=1,\;
   \braket{1}{1}=1.
\]
Thus \(\{\ket{0},\ket{1}\}\) is orthonormal, and
\(\norm{\ket{0}}=\norm{\ket{1}}=1\).

%%%%%%%%%%%%%%%%%%%%%%%%%%%%%%%%%%%%%%%%%%%%%%%%%%%%
\section*{3 \quad Probabilities from Amplitudes}

If \(\ket{\psi}=\alpha\ket{0}+\beta\ket{1}\), measuring in the
computational basis yields
\[
   P(0)=\abs{\alpha}^{2},\qquad
   P(1)=\abs{\beta}^{2}.
\]
The squared modulus converts a complex amplitude into a real,
non-negative probability, and normalisation
\(\abs{\alpha}^{2}+\abs{\beta}^{2}=1\) guarantees the outcomes exhaust
all possibilities.

%%%%%%%%%%%%%%%%%%%%%%%%%%%%%%%%%%%%%%%%%%%%%%%%%%%%
\section*{4 \quad Two Fundamental Single-Qubit Gates}

\subsection*{4.1 \quad Pauli–X Gate (Bit Flip)}
\[
   X =
   \begin{bmatrix}
     0 & 1\\
     1 & 0
   \end{bmatrix},
   \qquad
   X\ket{0}=\ket{1},\;
   X\ket{1}=\ket{0}.
\]

\subsection*{4.2 \quad Hadamard Gate (Superposition Maker)}
\[
   H =
   \frac{1}{\sqrt{2}}
   \begin{bmatrix}
     1 &  1\\
     1 & -1
   \end{bmatrix},
   \qquad
   H\ket{0}=\tfrac1{\sqrt2}(\ket{0}+\ket{1}),\;
   H\ket{1}=\tfrac1{\sqrt2}(\ket{0}-\ket{1}).
\]
\(H\) is its own inverse (\(H^{2}=I\)) and converts definite states into
equal superpositions (and vice versa).

%%%%%%%%%%%%%%%%%%%%%%%%%%%%%%%%%%%%%%%%%%%%%%%%%%%%
\section*{5 \quad Worked Examples}

\subsection*{Example 1: Apply \(X\) to \(\ket{0}\)}
\[
   X\ket{0} =
   \begin{bmatrix}0&1\\1&0\end{bmatrix}
   \begin{bmatrix}1\\0\end{bmatrix}
   =
   \begin{bmatrix}0\\1\end{bmatrix}
   =\ket{1}.
\]

\subsection*{Example 2: Apply \(H\) to \(\ket{0}\)}
\[
   H\ket{0}
   =
   \frac1{\sqrt2}
   \begin{bmatrix}1&1\\1&-1\end{bmatrix}
   \begin{bmatrix}1\\0\end{bmatrix}
   =
   \frac1{\sqrt2}
   \begin{bmatrix}1\\1\end{bmatrix}
   =
   \tfrac1{\sqrt2}(\ket{0}+\ket{1}).
\]

\subsection*{Example 3: Apply \(H\) to \(\ket{1}\)}
\[
   H\ket{1}
   =
   \frac1{\sqrt2}
   \begin{bmatrix}1&1\\1&-1\end{bmatrix}
   \begin{bmatrix}0\\1\end{bmatrix}
   =
   \frac1{\sqrt2}
   \begin{bmatrix}1\\-1\end{bmatrix}
   =
   \tfrac1{\sqrt2}(\ket{0}-\ket{1}).
\]

\subsection*{Example 4: Verify \(H^{2}=I\)}
\[
  H^{2}
  =
  \Bigl(\tfrac1{\sqrt2}\Bigr)^{\!2}
  \begin{bmatrix}1&1\\1&-1\end{bmatrix}^{2}
  =
  \frac12
  \begin{bmatrix}
     2 & 0\\
     0 & 2
  \end{bmatrix}
  =
  \begin{bmatrix}1&0\\0&1\end{bmatrix}=I_{2}.
\]

\paragraph{Guided exercise.}
Show that \(HXH=Z\), where
\(Z=\begin{bmatrix}1&0\\0&-1\end{bmatrix}\), and evaluate
\(HXH\ket{0}\) and \(XHX\ket{0}\),
noting the phase difference.

%%%%%%%%%%%%%%%%%%%%%%%%%%%%%%%%%%%%%%%%%%%%%%%%%%%%
\section*{6 \quad Key Takeaways}

\begin{enumerate}
  \item A qubit is a length-1 complex vector.
  \item \textbf{Orthonormal} basis vectors are perpendicular
        (\(\braket{u}{v}=0\)) \emph{and} length-1
        (\(\braket{u}{u}=1\), \(\norm{\ket{u}}=1\)).
  \item Conjugate transpose (\(^{\dagger}\)) means transpose + complex
        conjugation.
  \item Probabilities come from squared moduli of amplitudes.
  \item Single-qubit gates are \(2\times2\) unitary matrices.
\end{enumerate}

\end{document}
