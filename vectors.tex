\documentclass{article}
\usepackage[utf8]{inputenc} % Handle UTF-8 encoding
\usepackage{amsmath, amssymb, tikz, geometry, multicol}
\usetikzlibrary{calc}
\geometry{margin=0.15in}
\tolerance=1000

\begin{document}

% Start of the document with two columns
\begin{multicols}{2}

\section*{Vector Notation and Representation}

\subsection*{Unit Vector Notation}

Vectors can be represented in different notations:

- \textbf{Angle Bracket Notation}:

  A vector \(\vec{V}\) in two dimensions can be written as:

  \[
  \vec{V} = \langle V_x, V_y \rangle
  \]

- \textbf{Column Vector Notation}:

  \[
  \vec{V} = \begin{bmatrix} V_x \\ V_y \end{bmatrix}
  \]

- \textbf{Cartesian Coordinates}:

  Using the unit vectors \(\hat{i}\) and \(\hat{j}\):

  \[
  \vec{V} = V_x \hat{i} + V_y \hat{j}
  \]

Where:

- \( V_x \) is the component of the vector along the \( x \)-axis.
- \( V_y \) is the component of the vector along the \( y \)-axis.
- \( \hat{i} \) and \( \hat{j} \) are unit vectors in the \( x \) and \( y \) directions, respectively.

\subsection*{Zero-Length and Parallel Vectors}

- \textbf{Zero-Length Vector (Zero Vector)}:

  A vector with zero magnitude and no specific direction.

  \[
  \vec{0} = \langle 0, 0 \rangle
  \]

  \textbf{Properties:}

  - Adding the zero vector to any vector leaves the original vector unchanged.

    \[
    \vec{V} + \vec{0} = \vec{V}
    \]

- \textbf{Parallel Vectors}:

  Two vectors are parallel if they have the same or opposite directions.

  \textbf{Conditions for Parallelism:}

  - Vectors \(\vec{A}\) and \(\vec{B}\) are parallel if:

    \[
    \vec{A} = k \vec{B}
    \]

    where \( k \) is a scalar constant.

    - If \( k > 0 \), vectors are in the same direction.
    - If \( k < 0 \), vectors are in opposite directions.

  - \textbf{Component Proportionality:}

    \[
    \frac{A_x}{B_x} = \frac{A_y}{B_y}
    \]

    If the ratios of the corresponding components are equal (and finite), the vectors are parallel.

\subsection*{Position Vectors and Vector Difference}

If \( A \) and \( B \) are points in the Cartesian plane with position vectors \( \vec{a} \) and \( \vec{b} \) respectively, then the vector from point \( A \) to point \( B \) is given by:

\[
\vec{AB} = \vec{b} - \vec{a}
\]

This represents the displacement from point \( A \) to point \( B \).

\textbf{Example:}

Let point \( A \) have coordinates \( (2, 3) \) and point \( B \) have coordinates \( (5, 7) \).

- Write the position vectors:

  \[
  \vec{a} = \begin{bmatrix} 2 \\ 3 \end{bmatrix}, \quad \vec{b} = \begin{bmatrix} 5 \\ 7 \end{bmatrix}
  \]

- Compute \( \vec{AB} \):

  \[
  \vec{AB} = \vec{b} - \vec{a} = \begin{bmatrix} 5 \\ 7 \end{bmatrix} - \begin{bmatrix} 2 \\ 3 \end{bmatrix} = \begin{bmatrix} 5 - 2 \\ 7 - 3 \end{bmatrix} = \begin{bmatrix} 3 \\ 4 \end{bmatrix}
  \]

- The vector from \( A \) to \( B \) is \( \vec{AB} = \begin{bmatrix} 3 \\ 4 \end{bmatrix} \).

- The magnitude of \( \vec{AB} \) is:

  \[
  |\vec{AB}| = \sqrt{(3)^2 + (4)^2} = \sqrt{9 + 16} = \sqrt{25} = 5
  \]

- The direction (angle) of \( \vec{AB} \) with respect to the horizontal axis is:

  \[
  \theta = \arctan\left( \dfrac{4}{3} \right) \approx 53.13^\circ
  \]

\section*{Vector Decomposition and Addition}

\subsection*{Decomposing Vectors into Components}

Any vector \(\vec{V}\) can be decomposed into its horizontal and vertical components using trigonometric functions.

Given a vector of magnitude \( V \) at an angle \( \theta \) with respect to the horizontal axis:

\[
\begin{aligned}
V_x &= V \cos \theta \\
V_y &= V \sin \theta
\end{aligned}
\]

\begin{center}
\begin{tikzpicture}[scale=1]
    % Draw axes
    \draw[->] (-0.5,0) -- (5,0) node[right] {$x$};
    \draw[->] (0,-0.5) -- (0,3) node[above] {$y$};
    % Draw vector V
    \draw[->, thick] (0,0) -- (4,2) node[midway, above] {$\vec{V}$};
    % Components
    \draw[dashed] (4,0) -- (4,2);
    \draw[dashed] (0,2) -- (4,2);
    % Labels
    \draw[->] (0,0) -- (4,0) node[midway, below] {$V_x$};
    \draw[->] (0,0) -- (0,2) node[midway, left] {$V_y$};
    % Angle
    \draw (0.7,0) arc (0:26.57:0.7cm);
    \node at (1,0.3) {$\theta$};
\end{tikzpicture}
\end{center}

\textbf{Example:} A vector has a magnitude of \( V = 5 \) units and makes an angle of \( \theta = 30^\circ \) with the horizontal axis. Find its components.

\[
\begin{aligned}
V_x &= V \cos \theta = 5 \cos 30^\circ = 5 \times \dfrac{\sqrt{3}}{2} = \dfrac{5\sqrt{3}}{2} \approx 4.33 \\
V_y &= V \sin \theta = 5 \sin 30^\circ = 5 \times \dfrac{1}{2} = \dfrac{5}{2} = 2.5
\end{aligned}
\]

\subsection*{Adding Vectors Using Components}

To add two vectors, decompose each vector into its components, add the corresponding components, and then recombine the resultant components to find the resultant vector.

Given two vectors \(\vec{A}\) and \(\vec{B}\):

\[
\begin{aligned}
A_x &= A \cos \alpha \\
A_y &= A \sin \alpha \\
B_x &= B \cos \beta \\
B_y &= B \sin \beta
\end{aligned}
\]

The resultant vector \(\vec{R} = \vec{A} + \vec{B}\) has components:

\[
\begin{aligned}
R_x &= A_x + B_x \\
R_y &= A_y + B_y
\end{aligned}
\]

The magnitude and direction of \(\vec{R}\) are:

\[
\begin{aligned}
R &= \sqrt{R_x^2 + R_y^2} \\
\theta_R &= \arctan\left( \dfrac{R_y}{R_x} \right)
\end{aligned}
\]

\textbf{Example:} Add vector \(\vec{A}\) of magnitude \( A = 10 \) units at \( \alpha = 45^\circ \) and vector \(\vec{B}\) of magnitude \( B = 8 \) units at \( \beta = 120^\circ \).

\textit{Step 1: Decompose vectors into components.}

\[
\begin{aligned}
A_x &= 10 \cos 45^\circ = 10 \times \dfrac{\sqrt{2}}{2} = 5\sqrt{2} \approx 7.07 \\
A_y &= 10 \sin 45^\circ = 10 \times \dfrac{\sqrt{2}}{2} = 5\sqrt{2} \approx 7.07 \\
B_x &= 8 \cos 120^\circ = 8 \times \left( -\dfrac{1}{2} \right) = -4 \\
B_y &= 8 \sin 120^\circ = 8 \times \dfrac{\sqrt{3}}{2} = 4\sqrt{3} \approx 6.93
\end{aligned}
\]

\textit{Step 2: Add components.}

\[
\begin{aligned}
R_x &= A_x + B_x = 7.07 + (-4) = 3.07 \\
R_y &= A_y + B_y = 7.07 + 6.93 = 14
\end{aligned}
\]

\textit{Step 3: Calculate resultant magnitude and direction.}

\[
\begin{aligned}
R &= \sqrt{(R_x)^2 + (R_y)^2} = \sqrt{(3.07)^2 + (14)^2} \approx \sqrt{9.42 + 196} \approx \sqrt{205.42} \approx 14.33 \\
\theta_R &= \arctan\left( \dfrac{14}{3.07} \right) \approx 77.5^\circ
\end{aligned}
\]

So, the resultant vector has a magnitude of approximately \( 14.33 \) units and makes an angle of \( 77.5^\circ \) with the horizontal axis.

\subsection*{Summary}

- Decompose vectors into their \( x \) and \( y \) components using \( \cos \) and \( \sin \) functions.
- Add the corresponding components to find the resultant components.
- Calculate the magnitude and direction of the resultant vector from its components.

\end{multicols}

\end{document}
