\documentclass{article}
\usepackage[utf8]{inputenc}
\usepackage{amsmath, amssymb, tikz, geometry, multicol}
\geometry{margin=0.25in}

\begin{document}

%=== First Spread: Compound Rules & Substitution ===
\begin{multicols}{2}

%— Left Column: Differentiation —
\section*{Differentiation}

\subsection*{Definition}
The derivative of $f(x)$ at $x$ is
\[
f'(x) = \lim_{h\to 0}\frac{f(x+h)-f(x)}{h}.
\]

\subsection*{Product Rule}
If $f(x)=u(x)\,v(x)$, then
\[
f'(x)=u'(x)\,v(x)+u(x)\,v'(x).
\]

\subsection*{Quotient Rule}
If $f(x)=\dfrac{u(x)}{v(x)}$, then
\[
f'(x)=\frac{u'(x)\,v(x)-u(x)\,v'(x)}{[v(x)]^2}.
\]

\subsection*{Chain Rule}
If $f(x)=h(g(x))$, then
\[
f'(x)=h'(g(x))\;\cdot\;g'(x).
\]

\subsection*{Implicit Differentiation}
Given $F(x,y)=0$, differentiate both sides w.r.t. $x$, treating $y=y(x)$:
\[
F_x + F_y\,y'=0
\quad\Longrightarrow\quad
y'=-\frac{F_x}{F_y}.
\]
\emph{Example:} From $x^2+y^2=1$, $2x+2y\,y'=0\Rightarrow y'=-\tfrac{x}{y}.$

\subsection*{Differentiation by Substitution}
To differentiate $f(x)=H(u(x))$, set $u=u(x)$:
\[
\frac{df}{dx}=H'(u)\,\frac{du}{dx}.
\]
\emph{Example:} If $f(x)=(2x^3+x)^4$, then $u=2x^3+x$, $du= (6x^2+1) dx$, so
\[
f' = 4u^3 \cdot (6x^2+1) = 4(2x^3 + x)^3(6x^2+1).
\]

\columnbreak

%— Right Column: Integration —
\section*{Integration}

\subsection*{Definition}
\textbf{Definite integral:}
\[
\int_a^b f(x)dx=\lim_{n\to\infty}\sum_{i=1}^n f(x_i^*)\,\Delta x,
\quad \Delta x=\frac{b-a}{n}.
\]

\textbf{Indefinite integral:}
\[
\int f(x)dx=F(x)+C,
\quad F'(x)=f(x).
\]

\subsection*{Linearity Rules}
\[
\int\bigl[f(x)+g(x)\bigr]dx=\int f(x)dx+\int g(x)dx,
\]
\[
\int c\,f(x)dx=c\int f(x)dx.
\]

\subsection*{Implicit Integration}
For $y'=\frac{g(x)}{h(y)}$, separate:
\[
h(y)\,dy=g(x)dx\quad\Longrightarrow\quad
\int h(y)\,dy=\int g(x)dx + C.
\]

\subsection*{Integration by Substitution}
Let $u=g(x)$ so $du=g'(x)\,dx$. Then
\[
\int f(g(x))\,g'(x)\,dx=\int f(u)\,du.
\]
\emph{Example:} $\int x e^{x^2}dx$: with $u=x^2$, $du=2x dx$, gives $\tfrac12\int e^u du=\tfrac12 e^u + C$.

\subsection*{Definite Integration by Substitution}
To compute $\int_a^b f(g(x))g'(x)\,dx$, set $u=g(x)$ and update limits accordingly:
\[
\int_{x=a}^{x=b} f(g(x))g'(x)\,dx = \int_{u=g(a)}^{u=g(b)} f(u)\,du.
\]

% Table for limit conversion
\begin{center}
\begin{tabular}{c|cc}
Variable & Lower limit & Upper limit \\
\hline
$x$ & $a$ & $b$ \\
$u=g(x)$ & $g(a)$ & $g(b)$
\end{tabular}
\end{center}

\emph{Example:}
Compute
\[
\int_{0}^{\frac{\pi}{4}} \sec^2 x\tan x\,dx.
\]
Set $u=\tan x$, so $du=\sec^2 x\,dx$.  Limits:
\[
x=0\implies u=\tan0=0,
\quad x=\tfrac{\pi}{4}\implies u=\tan\tfrac{\pi}{4}=1.
\]
Table of limits:
\begin{center}
\begin{tabular}{c|cc}
Variable & Lower limit & Upper limit \\
\hline
$x$ & $0$ & $\tfrac{\pi}{4}$ \\
$u$ & $0$ & $1$
\end{tabular}
\end{center}
Thus,
\[
\int_{0}^{\frac{\pi}{4}} \sec^2 x\tan x\,dx
=\int_{0}^{1} u\,du
=\Bigl[\tfrac{u^2}{2}\Bigr]_{0}^{1}
=\tfrac{1}{2}.
\]

\subsection*{Integration by Parts}
Based on the product rule: $(uv)' = u'v + uv'$, we get the formula:
\[
\int u\,dv = uv - \int v\,du.
\]
Choose $u$ and $dv$ from the integrand so that $du$ and $v$ are easier to compute.

\emph{Example:} $\int x e^x\,dx$: let $u = x$, $dv = e^x\,dx$.
Then $du = dx$, $v = e^x$, so:
\[
\int x e^x\,dx = x e^x - \int e^x\,dx = x e^x - e^x + C.
\]

\subsection*{Definite Integration by Parts}
Apply the same rule with limits:
\[
\int_a^b u\,dv = \left[uv\right]_a^b - \int_a^b v\,du.
\]

\emph{Example:} Compute $\int_0^1 x \ln(x+1)\,dx$.

Let $u = \ln(x+1)$, so $du = \frac{1}{x+1}\,dx$;  
let $dv = x\,dx$, so $v = \frac{x^2}{2}$. Then:
\[
\int_0^1 x\ln(x+1)\,dx
= \left[\frac{x^2}{2} \ln(x+1)\right]_0^1 - \int_0^1 \frac{x^2}{2(x+1)}\,dx.
\]

\end{multicols}

\newpage
%=== Page 1: Non‑Trigonometric Pairs ===
\newpage
%=== Page 1: Non-Trigonometric Pairs ===
\begin{multicols}{2}
\section*{Derivative \& Integral Pairs}

\begin{align*}
\\[10pt]
%— Constant \& Linear-Power Substitution —
\frac{d}{dx}C &= 0
&\int 0\,dx &= C
\\[6pt]
\frac{d}{dx}(a x + b)^n &= n\,a\,(a x + b)^{n-1}
&\int (a x + b)^n\,dx &= \frac{(a x + b)^{n+1}}{a(n+1)} + C
\\[10pt]
%— Generic Exponential Substitution —
\frac{d}{dx}e^{u(x)} &= u'(x)\,e^{u(x)}
&\int u'(x)\,e^{u(x)}\,dx &= e^{u(x)} + C
\\[6pt]
\frac{d}{dx}a^{u(x)} &= u'(x)\,a^{u(x)}\ln a
&\int u'(x)\,a^{u(x)}\,dx &= \frac{a^{u(x)}}{\ln a} + C
\\[10pt]
%— Generic Logarithmic Substitution —
\frac{d}{dx}\ln\lvert u(x)\rvert &= \frac{u'(x)}{u(x)}
&\int \frac{u'(x)}{u(x)}\,dx &= \ln\lvert u(x)\rvert + C
\\[6pt]
\frac{d}{dx}\log_a\lvert u(x)\rvert &= \frac{u'(x)}{u(x)\,\ln a}
&\int \frac{u'(x)}{u(x)\,\ln a}\,dx &= \log_a\lvert u(x)\rvert + C
\\[10pt]
%— Algebraic, Exponential \& Logarithmic —
\frac{d}{dx}x^n &= n\,x^{n-1}
&\int x^n\,dx &= \frac{x^{n+1}}{n+1} + C
\\[6pt]
\frac{d}{dx}e^x &= e^x
&\int e^x\,dx &= e^x + C
\\[6pt]
\frac{d}{dx}a^x &= a^x\ln a
&\int a^x\,dx &= \frac{a^x}{\ln a} + C
\\[6pt]
\frac{d}{dx}e^{p x+q} &= p\,e^{p x+q}
&\int e^{p x+q}\,dx &= \frac{1}{p}e^{p x+q} + C
\\[6pt]
\frac{d}{dx}a^{p x+q} &= p\,a^{p x+q}\ln a
&\int a^{p x+q}\,dx &= \frac{1}{p\ln a}\,a^{p x+q} + C
\\[6pt]
\frac{d}{dx}\ln x &= \tfrac1x
&\int \tfrac{1}{x}\,dx &= \ln\lvert x\rvert + C
\\[6pt]
\frac{d}{dx}\log_a x &= \frac{1}{x\ln a}
&\int \frac{1}{x\ln a}\,dx &= \log_a\lvert x\rvert + C
\\[6pt]
\frac{d}{dx}\ln\lvert a x + b\rvert &= \frac{a}{a x + b}
&\int \frac{1}{a x + b}\,dx &= \frac{1}{a}\,\ln\lvert a x + b\rvert + C
\\[10pt]
\end{align*}
\end{multicols}

\newpage

%=== Page 2: Trigonometric Pairs ===
\begin{multicols}{2}
\section*{Trigonometric Derivative \& Integral Pairs}

\begin{align*}
%— Circular Functions —
\frac{d}{dx}\sin x &= \cos x
&\int \sin x dx &= -\cos x + C
\\[6pt]
\frac{d}{dx}\cos x &= -\sin x
&\int \cos x dx &= \sin x + C
\\[6pt]
\frac{d}{dx}\tan x &= \sec^2 x
&\int \tan x dx &= -\ln\lvert\cos x\rvert + C
\\[6pt]
\frac{d}{dx}\cot x &= -\csc^2 x
&\int \cot x dx &= \ln\lvert\sin x\rvert + C
\\[6pt]
\frac{d}{dx}\sec x &= \sec x,\tan x
&\int \sec x dx &= \ln\bigl\lvert\sec x+\tan x\bigr\rvert + C
\\[6pt]
\frac{d}{dx}\csc x &= -\csc x,\cot x
&\int \csc x dx &= -\ln\bigl\lvert\csc x+\cot x\bigr\rvert + C
\\[10pt]
%— Compound Trigonometric Functions —
\frac{d}{dx}\tan x &= \sec^2 x
&\int \sec^2 x dx &= \tan x + C
\\[6pt]
\frac{d}{dx}\cot x &= -\csc^2 x
&\int \csc^2 x dx &= -\cot x + C
\\[6pt]
\frac{d}{dx}\sec x &= \sec x\tan x
&\int \sec x\tan x dx &= \sec x + C
\\[6pt]
\frac{d}{dx}\csc x &= -\csc x\cot x
&\int \csc x\cot x dx &= -\csc x + C
\\[10pt]
\end{align*}

\columnbreak

\begin{align*}
%— Inverse Circular —
\frac{d}{dx}\arcsin x &= \frac{1}{\sqrt{1-x^2}}
&\int \frac{1}{\sqrt{1-x^2}},dx &= \arcsin x + C
\\[6pt]
\frac{d}{dx}\arccos x &= -\frac{1}{\sqrt{1-x^2}}
&\int -\frac{1}{\sqrt{1-x^2}},dx &= \arccos x + C
\\[6pt]
\frac{d}{dx}\arctan x &= \frac{1}{1+x^2}
&\int \frac{1}{1+x^2},dx &= \arctan x + C
\\[6pt]
\frac{d}{dx}\frac{1}{k}\arctan kx &= \frac{1}{1+(kx)^2}
&\int \frac{1}{1+(kx)^2}dx &= \frac{1}{k}\arctan kx + C
\\[6pt]
\frac{d}{dx}\frac{1}{k}\arctan \frac{x}{k} &= \frac{1}{k^2+(x)^2}
&\int \frac{1}{k^2+(x)^2}dx &= \frac{1}{k}\arctan \frac{x}{k} + C
\\[6pt]
\frac{d}{dx}\operatorname{arccot} x &= -\frac{1}{1+x^2}
&\int -\frac{1}{1+x^2},dx &= \operatorname{arccot} x + C
\\[6pt]
\frac{d}{dx}\operatorname{arcsec} x &= \frac{1}{|x|\sqrt{x^2-1}}
&\int \frac{1}{|x|\sqrt{x^2-1}}dx &= \operatorname{arcsec} x + C
\\[6pt]
\frac{d}{dx}\operatorname{arccsc} x &= -\frac{1}{|x|\sqrt{x^2-1}}
&\int -\frac{1}{|x|\sqrt{x^2-1}}dx &= \operatorname{arccsc} x + C
\\[10pt]
%— Hyperbolic & Inverse —
\frac{d}{dx}\sinh x &= \cosh x
&\int \cosh xdx &= \sinh x + C
\\[6pt]
\frac{d}{dx}\cosh x &= \sinh x
&\int \sinh xdx &= \cosh x + C
\\[6pt]
\frac{d}{dx}\tanh x &= \operatorname{sech}^2 x
&\int \operatorname{sech}^2 xdx &= \tanh x + C
\\[6pt]
\frac{d}{dx}\coth x &= -\operatorname{csch}^2 x
&\int \operatorname{csch}^2 xdx &= -\coth x + C
\\[6pt]
\frac{d}{dx}\operatorname{arsinh} x &= \frac{1}{\sqrt{1+x^2}}
&\int \frac{1}{\sqrt{1+x^2}}dx &= \operatorname{arsinh} x + C
\\[6pt]
\frac{d}{dx}\operatorname{arcosh} x &= \frac{1}{\sqrt{x^2-1}}
&\int \frac{1}{\sqrt{x^2-1}}dx &= \operatorname{arcosh} x + C
\\[6pt]
\frac{d}{dx}\operatorname{artanh} x &= \frac{1}{1 - x^2}
&\int \frac{1}{1 - x^2}dx &= \operatorname{artanh} x + C
\\[6pt]
\frac{d}{dx}\operatorname{arsech} x &= -\frac{1}{x\sqrt{1 - x^2}}
&\int -\frac{1}{x\sqrt{1 - x^2}}dx &= \operatorname{arsech} x + C
\\[6pt]
\frac{d}{dx}\operatorname{arccsch} x &= -\frac{1}{|x|\sqrt{1 + x^2}}
&\int -\frac{1}{|x|\sqrt{1 + x^2}}dx &= \operatorname{arccsch} x + C
\end{align*}

\end{multicols}
\end{document}
