\documentclass[11pt]{article}
\usepackage{amsmath,amssymb}
\usepackage{graphicx}
\usepackage{listings}
\usepackage{hyperref}
\usepackage{geometry}
\geometry{margin=1in}

\title{From Beat Notes to Wavelet Decomposition:\\
A Self-Contained Walk Through Sines, Windows, and Classic Wavelets}
\author{}
\date{}

% ---------------------------------------------------
% Listings style
\lstset{
  basicstyle=\ttfamily\small,
  frame=single,
  columns=fullflexible,
  showstringspaces=false
}
% ---------------------------------------------------

\begin{document}
\maketitle

\tableofcontents
\bigskip

% ===================================================
\section{Combined Sine Waves and Beat Notes}
% ===================================================

\paragraph{Definition.}
Let \(f_1, f_2 \in \mathbb{R}_{>0}\) be two pitch frequencies in hertz (cycles per second).
A \emph{pure sinusoid} at frequency \(f\) is \( \sin(2\pi f t) \).
The constant \( \pi \) (Greek letter \emph{pi}) was first used for the circle ratio by
William Jones in 1706 and has since become universal for angular calculations.

\paragraph{Beat-note identity.}
Adding two pure sinusoids gives
\[
\sin(2\pi f_1 t) + \sin(2\pi f_2 t)
     = 2 \cos\!\Bigl(2\pi \frac{f_1 - f_2}{2}\, t\Bigr)\,
       \sin\!\Bigl(2\pi \frac{f_1 + f_2}{2}\, t\Bigr).
\]
The factor \( \cos(2\pi |f_1-f_2|t/2) \) modulates amplitude at the
\emph{beat frequency} \( |f_1 - f_2| \).  When \( f_1 \) and \( f_2 \) differ only
slightly, the amplitude rises and falls slowly; when they differ greatly,
no slow modulation is audible or visible in the waveform.

\begin{lstlisting}[language=Python,caption={One-second beat note at 110 Hz and 116 Hz}]

import numpy as np, matplotlib.pyplot as plt
from IPython.display import Audio

sr = 44100                         # sample rate (Hz)
f1, f2 = 110, 116                  # A2 and B♭2
t  = np.linspace(0, 1.0, sr, endpoint=False)
sig = np.sin(2*np.pi*f1*t) + np.sin(2*np.pi*f2*t)
sig = sig / np.max(np.abs(sig))    # normalise

plt.plot(t, sig)
plt.title("110 Hz + 116 Hz — beat ≈ 6 Hz")
plt.xlabel("time (s)"); plt.ylabel("amplitude")
plt.show()

Audio(sig, rate=sr)
\end{lstlisting}

\paragraph{Piano demonstration.}
Press \(A_2\) (110 Hz) together with \(B\flat_2\) (approximately 116 Hz).
A volume undulation at \(\approx 6\) Hz is heard; this matches the
\(\cos\)-modulation predicted above.

% ===================================================
\section{Need for Temporal Localisation and the Gaussian Window}
% ===================================================

\paragraph{Problem statement.}
Beat analysis reveals \emph{global} interference patterns, but real data
(speech syllables, seismic reflections, image edges) require knowing
\emph{when} a feature occurs.  A window function \(w(t)\) multiplies the
signal, confining analysis to a finite time span.

\paragraph{Gaussian window.}
\[
w(t)=\exp\!\Bigl(-\frac{t^{2}}{2\sigma^{2}}\Bigr).
\]
The Greek letter \( \sigma \) (\emph{sigma}) denotes \emph{standard
deviation}, introduced with that meaning by Karl Pearson in 1894.  Carl
Friedrich Gauss published this bell formula in 1809 while modelling
astronomical error distributions.

\paragraph{Heisenberg–Gabor limit.}
Werner Heisenberg (1927) proved for any square-integrable \(g(t)\)
\[
(\Delta t)(\Delta f)\ \ge\ \tfrac12,
\]
where \(\Delta t\) and \(\Delta f\) are root-mean-square spreads in time
and frequency.  Dennis Gabor (1946) applied the same inequality to signal
processing.  The Gaussian attains equality, giving the best possible
simultaneous concentration in both domains.

\begin{lstlisting}[language=Python,caption={Plot of Gaussian window with σ = 0.25 s}]
sigma = 0.25
t     = np.linspace(-1, 1, sr*2, endpoint=False)
g     = np.exp(-(t**2)/(2*sigma**2))

plt.plot(t, g)
plt.title(r"Gaussian $e^{-t^{2}/2\sigma^{2}}$  ($\sigma = 0.25\,$s)")
plt.xlabel("time (s)"); plt.ylabel("amplitude")
plt.show()
\end{lstlisting}

% ===================================================
\section{Euler’s Formula and the Argand Plane Rotation}
% ===================================================

Leonhard Euler (1748) established
\[
e^{i\theta} = \cos\theta + i\sin\theta,
\]
where \( \theta \) (Greek \emph{theta}, adopted by Euler for angles)
measures rotation in radians.  Jean-Robert Argand (1806) introduced the
geometric representation of complex numbers: the point
\( (\cos\theta,\sin\theta) \) is a unit vector making angle \( \theta \)
with the positive \(x\)-axis.

Define
\[
z(t) = w(t)\,e^{i\omega t},\qquad \omega = 2\pi f,
\]
with \( \omega \) (Greek \emph{omega}) chosen for angular frequency by
William Thomson (Lord Kelvin) c.\,1867.  Here \( |z(t)| = w(t) \) and the
argument of \(z(t)\) advances at \(\omega\).

\paragraph{Animated Argand arrow.}
Insert the following in a Colab cell; it shows the arrow growing and
shrinking with \(w(t)\) while rotating.

\begin{lstlisting}[language=Python,caption={Argand rotation with Gaussian magnitude}]
import matplotlib.pyplot as plt
from matplotlib.animation import FuncAnimation
import numpy as np
from IPython.display import HTML

sigma, f = 0.25, 1.0
duration, fps = 2.0, 30
frames  = int(duration*fps)
ts      = np.linspace(-duration/2, duration/2, frames)
radii   = np.exp(-(ts**2)/(2*sigma**2))
angles  = 2*np.pi*f*ts

fig, ax = plt.subplots(figsize=(4,4))
ax.set_aspect("equal"), ax.axis("off")
ax.set_xlim(-1.1,1.1), ax.set_ylim(-1.1,1.1)
circle = plt.Circle((0,0), 1, color="lightgray", fill=False, ls="--")
ax.add_patch(circle)
line, = ax.plot([], [], lw=3, color="darkorange")

def update(i):
    r, th = radii[i], angles[i]
    x, y  = r*np.cos(th), r*np.sin(th)
    line.set_data([0,x],[0,y]); line.set_alpha(max(r,0.05))
    return line,
ani = FuncAnimation(fig, update, frames=frames, blit=True, interval=1000/fps)
HTML(ani.to_jshtml())
\end{lstlisting}

% ===================================================
\section{Gaussian-Windowed Cosine–Sine Pair (Morlet Packet)}
% ===================================================

\paragraph{Definition.}
\[
\psi_{\cos}(t)=w(t)\cos\omega t,\qquad
\psi_{\sin}(t)=w(t)\sin\omega t.
\]
Both phases are required: a local signal might begin at a maximum
(cosine-aligned) or at a zero-crossing (sine-aligned).

\begin{lstlisting}[language=Python,caption={Plot \& play real part of a Morlet packet}]
def morlet_real(t, f=440, sigma=0.05):
    w = np.exp(-(t**2)/(2*sigma**2))
    return w*np.cos(2*np.pi*f*t)

dur = 0.4
t   = np.linspace(-dur/2, dur/2, int(sr*dur), endpoint=False)
sig = morlet_real(t)

plt.plot(t, sig)
plt.title("Morlet (real)  440 Hz  σ=0.05 s")
plt.xlabel("time (s)"); plt.ylabel("amplitude"); plt.show()
Audio(sig/np.max(np.abs(sig)), rate=sr)
\end{lstlisting}

% ===================================================
\section{Classic Wavelet Forms and Their Origins}
% ===================================================

\subsection{Mexican-Hat (Ricker) Wavelet}
\[
\psi_{\text{MH}}(t)=\Bigl(1-\frac{t^{2}}{\sigma^{2}}\Bigr)
                    \exp\!\Bigl(-\frac{t^{2}}{2\sigma^{2}}\Bigr).
\]
\textbf{History.}  Norman Ricker (1953) devised this even-symmetric pulse
while investigating seismic reflection shapes.

\paragraph{Current use.}  Detecting circular blobs in images; spotting
P- and S-wave arrivals in modern seismology.

\subsection{Morlet Wavelet}
Complex Gaussian packet (1978–1984) by petroleum geophysicist
Jean Morlet, formalised with Alex Grossmann.  Enables the continuous
wavelet transform (CWT) now standard in EEG microstate analysis.

\subsection{Haar Wavelet}
Alfréd Haar (1909) introduced the first orthonormal step function for his
thesis on orthogonal series.  The mother wavelet is
\(\psi(t)=1\) on \([0,\tfrac12)\), \(-1\) on \([\tfrac12,1)\), \(0\) otherwise.
Widely used in lossless image codecs.

\subsection{Daubechies Wavelets}
Ingrid Daubechies (1987) created compact-support, smooth, orthogonal
wavelets; db4 uses filter
\(h=[0.4829629, 0.8365163, 0.2241439,-0.1294095]\).
Employed by the FBI for fingerprint compression and in numerical PDE
schemes.

% ===================================================
\section{First-Level Haar Discrete Wavelet Transform}
% ===================================================

For four samples \(x=[x_0,x_1,x_2,x_3]\),

\[
\text{averages}=\frac{x_0+x_1}{2},\;\frac{x_2+x_3}{2},\qquad
\text{details}=\frac{x_0-x_1}{2},\;\frac{x_2-x_3}{2}.
\]

\begin{lstlisting}[language=Python,caption={Manual 4-sample Haar DWT}]
x = np.array([7., 5., 1., 1.])
avg  = (x[0::2] + x[1::2]) / 2   # [6, 1]
detail = (x[0::2] - x[1::2]) / 2 # [1, 0]
coeffs = np.concatenate([avg, detail])
print(coeffs)   # -> [6. 1. 1. 0.]
\end{lstlisting}

% ===================================================
\section{Wavelet Decomposition Beyond Level 1}
% ===================================================

Apply the same average/detail split recursively to the coarse block,
doubling time resolution at each level.  For practical work use a vetted
library (e.g.\ PyWavelets), but every algorithm obeys the step illustrated above.

% ===================================================
\section*{Summary}

Starting from two audible sine waves we derived beat modulation, then
introduced Gaussian time-windows to control temporal locality, respecting
the Heisenberg–Gabor uncertainty limit.  Euler’s formula and the Argand
plane provided a geometric view of complex oscillations, motivating the
Gaussian-windowed cosine–sine pair (Morlet).  We reviewed historically
significant wavelets—Mexican-Hat, Morlet, Haar, Daubechies—and executed a
hand calculation of the first-level Haar transform.

All code samples can be typed directly into a Google Colab notebook and
will produce the stated plots and audio widgets.

\end{document}
