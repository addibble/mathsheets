\documentclass[12pt]{article}
\usepackage{amsmath,amssymb}
\usepackage{geometry}
\geometry{margin=1in}
\usepackage{graphicx}
\usepackage{fancyhdr}
\pagestyle{fancy}
\fancyhf{}
\rhead{Polynomials and Roots}
\lhead{Math Exploration}
\rfoot{Page \thepage}

\title{\vspace{-2cm}What Are Roots?}
\author{}
\date{}

\begin{document}
\maketitle

\section*{1. What Is a Root?}

A \textbf{root} (also called a \textbf{zero}) of a polynomial is a number that makes the polynomial equal to zero.

\bigskip

For example, if we have the polynomial:
\[
f(x) = x - 3,
\]
we can find the root by setting:
\[
f(x) = 0 \Rightarrow x - 3 = 0 \Rightarrow x = 3.
\]

So, \fbox{$x = 3$ is a root of $f(x)$}.

\bigskip

\textbf{Roots are where the graph of a function crosses the $x$-axis.} That means the output $y = 0$ at those points. In Scratch, you can graph this and look for where your curve hits $y = 0$.

\vspace{0.5cm}

\section*{2. The Zero Product Rule}

If
\[
(a)(b) = 0,
\]
then either:
\[
a = 0 \quad \text{or} \quad b = 0.
\]

This is called the \textbf{zero product rule}, and it helps us solve equations that are \textbf{factored}.

\bigskip

\textbf{Example:}

\[
(x - 2)(x + 5) = 0
\]

Apply the zero product rule:
\[
x - 2 = 0 \quad \text{or} \quad x + 5 = 0
\Rightarrow x = 2 \quad \text{or} \quad x = -5
\]

So, the roots are \fbox{$x = 2$ and $x = -5$}.

\bigskip

\section*{3. Try It Yourself!}

\vspace{0.2cm}
\begin{itemize}
  \item[(a)] What are the roots of $f(x) = (x - 4)(x + 1)$?
  \vspace{1.2cm}
  
  \item[(b)] What are the roots of $f(x) = x(x - 3)$?
  \vspace{1.2cm}
  
  \item[(c)] Factor and find the roots: $f(x) = x^2 - 9$
  \vspace{1.2cm}
  
  \item[(d)] Try a harder one: $f(x) = x^2 + 5x + 6$
  \vspace{1.2cm}
\end{itemize}

\section*{4. Graphing Roots in Scratch}

If you program a graphing calculator in Scratch, try this:

\begin{itemize}
  \item Plot $f(x) = x^2 - 4$
  \item Watch where the graph crosses the $x$-axis
  \item The graph hits $y = 0$ at $x = -2$ and $x = 2$ — those are the \textbf{roots}.
\end{itemize}

\bigskip

\textbf{Challenge:} Make your Scratch calculator \textbf{highlight} the x-axis in red and show where the function equals 0.

\vspace{0.5cm}

\section*{5. Think About It...}

\begin{itemize}
  \item Why do some graphs cross the x-axis twice, once, or not at all?
  \item What does it mean when a root happens twice? Try $f(x) = (x - 2)^2$
  \item Can a function have \textit{no real roots}? Try $f(x) = x^2 + 1$
\end{itemize}

\vspace{0.5cm}

\section*{6. Bonus Question (for Math Wizards)}

What are the roots of:
\[
f(x) = x^3 - 6x^2 + 11x - 6?
\]

Hint: Try factoring it step by step!

\vfill

\centering
\textit{Math is the art of finding patterns — and roots show us where patterns touch zero.}
\end{document}
