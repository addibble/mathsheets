\documentclass{article}
\usepackage[utf8]{inputenc}
\usepackage{amsmath, amssymb, tikz, geometry, multicol}
\geometry{margin=0.25in}
\setlength{\columnsep}{1cm} % Adjust column separation if needed

\begin{document}

\begin{multicols}{2}

\section*{Time Dilation in Special Relativity}

\subsection*{Lorentz Factor}

The Lorentz factor (\(\gamma\)) describes how time dilates for an object moving at velocity \(v\) relative to an observer:

\[
\gamma = \frac{1}{\sqrt{1 - \left( \dfrac{v}{c} \right)^2 }}
\]

where:
- \( v \) = velocity of the moving object,
- \( c \) = speed of light (\( c = 299{,}792{,}458 \, \text{m/s} \)).

\subsection*{Time Dilation Formula}

The relationship between time intervals measured in different frames:

\[
\Delta t = \gamma \Delta t'
\]

- \( \Delta t' \) = time interval measured in the \textbf{moving frame} (also known as \emph{proper time}),
- \( \Delta t \) = time interval measured in the \textbf{stationary frame}.

\subsection*{Deriving Time Difference}

To find the time difference (\( \Delta \tau \)) between the two frames:

\[
\Delta \tau = \Delta t - \Delta t' = \gamma \Delta t' - \Delta t' = \Delta t' (\gamma - 1)
\]

\subsection*{Solving for Time in Moving Frame}

Given a desired time difference (\( \Delta \tau \)):

\[
\Delta t' = \frac{\Delta \tau}{\gamma - 1}
\]

\subsection*{Calculations at Various Speeds}

\subsubsection*{Case 1: \( v = 80\% \, c \)}

\textbf{Calculate Lorentz Factor:}

For \( v = 0.8c \):

\[
\gamma = \frac{1}{\sqrt{1 - (0.8)^2}} = \frac{1}{\sqrt{0.36}} = \frac{1}{0.6} \approx 1.6667
\]

\textbf{Find Time in Moving Frame for \(\Delta \tau = 60\, \text{s}\):}

\[
\Delta t' = \frac{60\, \text{s}}{1.6667 - 1} = \frac{60\, \text{s}}{0.6667} \approx 90\, \text{s}
\]

\textbf{Calculate Time in Stationary Frame:}

\[
\Delta t = \gamma \Delta t' = 1.6667 \times 90\, \text{s} = 150\, \text{s}
\]

\textbf{Verify Time Difference:}

\[
\Delta \tau = \Delta t - \Delta t' = 150\, \text{s} - 90\, \text{s} = 60\, \text{s}
\]

\subsubsection*{Case 2: \( v = 90\% \, c \)}

\textbf{Calculate Lorentz Factor:}

For \( v = 0.9c \):

\[
\gamma = \frac{1}{\sqrt{1 - (0.9)^2}} = \frac{1}{\sqrt{0.19}} \approx 2.2942
\]

\textbf{Find Time in Moving Frame for \(\Delta \tau = 60\, \text{s}\):}

\[
\Delta t' = \frac{60\, \text{s}}{2.2942 - 1} \approx 46.3615\, \text{s}
\]

\textbf{Calculate Time in Stationary Frame:}

\[
\Delta t = \gamma \Delta t' \approx 2.2942 \times 46.3615\, \text{s} \approx 106.3615\, \text{s}
\]

\textbf{Verify Time Difference:}

\[
\Delta \tau = \Delta t - \Delta t' = 106.3615\, \text{s} - 46.3615\, \text{s} = 60\, \text{s}
\]

\subsubsection*{Case 3: \( v = 99\% \, c \)}

\textbf{Calculate Lorentz Factor:}

For \( v = 0.99c \):

\[
\gamma = \frac{1}{\sqrt{1 - (0.99)^2}} \approx 7.0888
\]

\textbf{Find Time in Moving Frame for \(\Delta \tau = 60\, \text{s}\):}

\[
\Delta t' = \frac{60\, \text{s}}{7.0888 - 1} \approx 9.8581\, \text{s}
\]

\textbf{Calculate Time in Stationary Frame:}

\[
\Delta t = \gamma \Delta t' \approx 7.0888 \times 9.8581\, \text{s} \approx 69.8581\, \text{s}
\]

\textbf{Verify Time Difference:}

\[
\Delta \tau = \Delta t - \Delta t' = 69.8581\, \text{s} - 9.8581\, \text{s} = 60\, \text{s}
\]

\columnbreak

\section*{Energy Requirements}

\subsection*{Relativistic Kinetic Energy}

The kinetic energy (\( KE \)) required to accelerate an object to a relativistic speed is:

\[
KE = (\gamma - 1) m c^2
\]

where:
- \( m \) = rest mass of the object,
- \( \gamma \) = Lorentz factor,
- \( c \) = speed of light.

\subsection*{Calculating Energy for the DeLorean}

Assuming the DeLorean has a mass \( m = 1{,}230\, \text{kg} \) (approximate mass of a DeLorean car).

\subsubsection*{Case 1: \( v = 80\% \, c \)}

\textbf{Calculate Lorentz Factor:}

Already calculated: \( \gamma \approx 1.6667 \)

\textbf{Calculate Kinetic Energy:}

\[
KE = (\gamma - 1) m c^2 = (1.6667 - 1) \times 1{,}230\, \text{kg} \times (299{,}792{,}458\, \text{m/s})^2
\]

\[
KE \approx 0.6667 \times 1{,}230\, \text{kg} \times (8.9875 \times 10^{16}\, \text{m}^2/\text{s}^2)
\]

\[
KE \approx 0.6667 \times 1{,}230\, \text{kg} \times 8.9875 \times 10^{16}\, \text{J/kg}
\]

\[
KE \approx 0.6667 \times 1.1065 \times 10^{20}\, \text{J} \approx 7.3765 \times 10^{19}\, \text{J}
\]

\textbf{Result:}

Approximately \( 7.38 \times 10^{19} \) joules of energy are required.

\subsubsection*{Case 2: \( v = 90\% \, c \)}

\textbf{Calculate Lorentz Factor:}

Already calculated: \( \gamma \approx 2.2942 \)

\textbf{Calculate Kinetic Energy:}

\[
KE = (2.2942 - 1) \times 1{,}230\, \text{kg} \times c^2
\]

\[
KE \approx 1.2942 \times 1{,}230\, \text{kg} \times 8.9875 \times 10^{16}\, \text{J/kg}
\]

\[
KE \approx 1.2942 \times 1.1065 \times 10^{20}\, \text{J} \approx 1.4323 \times 10^{20}\, \text{J}
\]

\textbf{Result:}

Approximately \( 1.43 \times 10^{20} \) joules of energy are required.

\subsubsection*{Case 3: \( v = 99\% \, c \)}

\textbf{Calculate Lorentz Factor:}

Already calculated: \( \gamma \approx 7.0888 \)

\textbf{Calculate Kinetic Energy:}

\[
KE = (7.0888 - 1) \times 1{,}230\, \text{kg} \times c^2
\]

\[
KE \approx 6.0888 \times 1{,}230\, \text{kg} \times 8.9875 \times 10^{16}\, \text{J/kg}
\]

\[
KE \approx 6.0888 \times 1.1065 \times 10^{20}\, \text{J} \approx 6.7389 \times 10^{20}\, \text{J}
\]

\textbf{Result:}

Approximately \( 6.74 \times 10^{20} \) joules of energy are required.

\subsection*{Understanding the Energy Scale}

For perspective:

- The total annual energy consumption of the entire world is about \( 5 \times 10^{20} \) joules.
- Accelerating the DeLorean to \( 99\% \, c \) requires more energy than the world's annual energy consumption.

\subsection*{Instructions to Calculate Energy Required}

To calculate the energy required to accelerate an object to a given velocity:

\begin{enumerate}
    \item \textbf{Calculate Lorentz Factor} (\( \gamma \)):
    \[
    \gamma = \frac{1}{\sqrt{1 - \left( \dfrac{v}{c} \right)^2 }}
    \]
    \item \textbf{Compute Kinetic Energy} (\( KE \)):
    \[
    KE = (\gamma - 1) m c^2
    \]
    \item \textbf{Plug in the Mass and Speed of Light}:
    \begin{itemize}
        \item \( m \) = mass of the object (in kg),
        \item \( c \) = \( 299{,}792{,}458\, \text{m/s} \).
    \end{itemize}
    \item \textbf{Calculate and Interpret the Result}:
    \begin{itemize}
        \item The result is in joules (J).
        \item Compare with known energy scales for perspective.
    \end{itemize}
\end{enumerate}

\subsection*{Example Calculation}

Given:
- \( v = 95\% \, c \)
- \( m = 1{,}230\, \text{kg} \)

\textbf{Step 1: Calculate Lorentz Factor}

\[
\gamma = \frac{1}{\sqrt{1 - (0.95)^2}} \approx 3.2026
\]

\textbf{Step 2: Compute Kinetic Energy}

\[
KE = (3.2026 - 1) \times 1{,}230\, \text{kg} \times c^2
\]

\[
KE \approx 2.2026 \times 1{,}230\, \text{kg} \times 8.9875 \times 10^{16}\, \text{J/kg}
\]

\[
KE \approx 2.2026 \times 1.1065 \times 10^{20}\, \text{J} \approx 2.4361 \times 10^{20}\, \text{J}
\]

\textbf{Result:}

Approximately \( 2.44 \times 10^{20} \) joules of energy are required.

\section*{Key Equations Summary}

\subsection*{Lorentz Factor}

\[
\gamma = \frac{1}{\sqrt{1 - \left( \dfrac{v}{c} \right)^2 }}
\]

\subsection*{Time Dilation}

\[
\Delta t = \gamma \Delta t'
\]

\subsection*{Time Difference}

\[
\Delta \tau = \Delta t - \Delta t' = \Delta t' (\gamma - 1)
\]

\subsection*{Solving for Time in Moving Frame}

\[
\Delta t' = \frac{\Delta \tau}{\gamma - 1}
\]

\subsection*{Relativistic Kinetic Energy}

\[
KE = (\gamma - 1) m c^2
\]

\end{multicols}

\end{document}
