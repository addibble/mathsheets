\documentclass[12pt]{article}
\usepackage{amsmath}
\usepackage{graphicx}
\usepackage{physics}
\usepackage{hyperref}
\usepackage{fancyhdr}
\usepackage{enumitem}
\usepackage{geometry}
\geometry{margin=1in}

\pagestyle{fancy}
\fancyhead[L]{}
\fancyhead[C]{Photon Energy and Sunscreen}
\fancyhead[R]{UV Physics}

\title{\bfseries Photon Energy and Sunscreen Molecule Absorption Worksheet}
\author{}
\date{}

\begin{document}

\maketitle

\section*{Objective}
Calculate the energy of photons at different wavelengths and identify which sunscreen molecules are capable of absorbing or blocking them.

\section*{Photon Energy Formula}

The energy of a single photon depends on its wavelength and is calculated using:

\[
E = \frac{hc}{\lambda}
\]

Where:
\begin{itemize}
    \item $E$ is energy in joules (J)
    \item $h = 6.626 \times 10^{-34}\ \text{J}\cdot\text{s}$ (Planck's constant)
    \item $c = 3.00 \times 10^8\ \text{m/s}$ (speed of light)
    \item $\lambda$ is the wavelength in meters (m)
\end{itemize}

To express energy in electronvolts (eV), we divide by the elementary charge:

\[
E(\text{eV}) = \frac{hc}{\lambda} \cdot \frac{1}{e}
\quad \text{where } e = 1.602 \times 10^{-19}\ \text{C}
\]

Substituting numerical values and converting nanometers to meters:

\[
E(\text{eV}) = \frac{(6.626 \times 10^{-34})(3.00 \times 10^8)}{(1.602 \times 10^{-19})(\lambda \times 10^{-9})}
\]

\[
E(\text{eV}) = \frac{1.986 \times 10^{-25}}{1.602 \times 10^{-28} \cdot \lambda}
= \frac{1240}{\lambda\ (\text{nm})}
\]

\noindent
\textbf{Therefore, the simplified formula:}

\[
E(\text{eV}) = \frac{1240}{\lambda\ (\text{nm})}
\]

gives the energy of a photon in electronvolts when the wavelength is expressed in nanometers.

\section*{Clarifying the Constant \( e \): Charge vs. Euler’s Number}

In the formula for photon energy in electronvolts:

\[
E(\text{eV}) = \frac{hc}{\lambda} \cdot \frac{1}{e}
\]

the symbol \( e \) refers to the \textbf{elementary charge}, not Euler’s number.

\begin{itemize}
    \item \( e = 1.602 \times 10^{-19}\ \text{C} \) is the electric charge of a single electron or proton.
    \item It is used to convert energy from joules (J) to electronvolts (eV), since:
    \[
    1\ \text{eV} = 1.602 \times 10^{-19}\ \text{J}
    \]
\end{itemize}

\noindent
In contrast, Euler’s number \( e \approx 2.718 \) is a mathematical constant used in exponential functions, and is not involved in photon energy calculations.

\vspace{0.5em}
\noindent
\textbf{Remember:} In physics, always check whether a symbol like \( e \) refers to a physical constant or a mathematical one based on the context.

\section*{Types of UV Light}

\begin{center}
\begin{tabular}{|c|c|}
\hline
\textbf{UV Band} & \textbf{Wavelength Range (nm)} \\
\hline
UV-C & 100–280 \\
UV-B & 280–315 \\
UV-A & 315–400 \\
\hline
\end{tabular}
\end{center}

\section*{Sunscreen Molecule Absorption Ranges}

\begin{center}
\begin{tabular}{|l|c|}
\hline
\textbf{Molecule} & \textbf{Absorption Range (nm)} \\
\hline
Oxybenzone & 270–350 \\
Avobenzone & 310–400 \\
Octinoxate & 280–320 \\
Zinc oxide & 280–400 (broad-spectrum) \\
Titanium dioxide & 290–400 (broad-spectrum) \\
\hline
\end{tabular}
\end{center}

\section*{Exercise 1: Photon Energies}

Use the formula \( E(\text{eV}) = \frac{1240}{\lambda} \) to fill in the missing energies.

\begin{center}
\begin{tabular}{|c|c|}
\hline
Wavelength (nm) & Energy (eV) \\
\hline
250 & \underline{\hspace{3cm}} \\
280 & \underline{\hspace{3cm}} \\
310 & \underline{\hspace{3cm}} \\
350 & \underline{\hspace{3cm}} \\
400 & \underline{\hspace{3cm}} \\
500 & \underline{\hspace{3cm}} \\
\hline
\end{tabular}
\end{center}

\section*{Exercise 2: Match Wavelengths to Absorbing Molecules}

For each wavelength below, list all the sunscreen molecules from the table that are capable of absorbing or blocking that wavelength.

\begin{itemize}
    \item 250 nm: \underline{\hspace{10cm}}
    \item 280 nm: \underline{\hspace{10cm}}
    \item 310 nm: \underline{\hspace{10cm}}
    \item 350 nm: \underline{\hspace{10cm}}
    \item 400 nm: \underline{\hspace{10cm}}
    \item 500 nm: \underline{\hspace{10cm}}
\end{itemize}

\section*{Discussion Questions}

\begin{enumerate}
    \item Why can't red light (e.g., 600–700 nm) cause sunburn, even though it's bright?
    \item Which molecules offer the best protection against UV-B? Against UV-A?
    \item What does “broad-spectrum” mean in sunscreen labels?
    \item Why is UV-C not usually a concern in daily sunscreen use?
\end{enumerate}

\end{document}
