\documentclass[12pt]{article}
\usepackage{amsmath}
\usepackage{graphicx}
\usepackage{physics}
\usepackage{hyperref}
\usepackage{fancyhdr}
\usepackage{enumitem}
\usepackage{geometry}
\geometry{margin=1in}

\pagestyle{fancy}
\fancyhead[L]{}
\fancyhead[C]{The Photoelectric Effect}
\fancyhead[R]{Einstein, 1905}

\title{\bfseries The Photoelectric Effect: Experimental Evidence for the Particle Nature of Light}
\author{}
\date{}

\begin{document}

\maketitle

\section*{Objective}
To demonstrate that light exhibits particle-like behavior by using a simple electroscope and ultraviolet light to reproduce the essential findings of the photoelectric effect, as explained by Albert Einstein in 1905.

\section*{Background}
Classically, light was understood as a continuous electromagnetic wave. Wave theory predicted that increasing light intensity should increase the energy delivered to a surface, regardless of color (frequency). However, this prediction fails to explain observed phenomena such as the photoelectric effect.

In 1905, Albert Einstein proposed that light consists of discrete packets of energy called \emph{quanta} or \emph{photons}. He wrote:

\begin{quote}
    ``Energy of light is not distributed continuously over wavefronts, but rather consists of energy quanta of magnitude $E = hf$, where $h$ is Planck's constant and $f$ is the frequency of light.'' \\
    \hfill -- Einstein, \emph{Annalen der Physik}, 1905
\end{quote}

This model explained why electrons are only emitted from a material when the incoming light has a frequency above a certain threshold, regardless of intensity.

\section*{Einstein's Predictions}
Einstein's photon theory made the following predictions, in direct contrast with classical wave theory:

\begin{enumerate}[label=(\alph*)]
    \item Electrons are ejected only if the light frequency $f$ exceeds a material-specific threshold $f_0$.
    \item The kinetic energy of emitted electrons increases linearly with frequency, $K_{\text{max}} = hf - \phi$, where $\phi$ is the work function.
    \item There is no time delay in electron emission, even at low light intensities.
    \item Increasing intensity at sub-threshold frequencies produces no emission.
\end{enumerate}

\section*{Materials (DIY Setup)}
\begin{itemize}
    \item Aluminum foil
    \item Clear plastic or glass cup/jar
    \item Thin strips of foil (electroscope leaves)
    \item Plastic comb or rod
    \item Wool cloth or sweater
    \item UV flashlight (365--395 nm, ideally with visible-blocking filter)
    \item Glass from picture frame (UV-blocking)
    \item Optional: visible LED flashlight (for comparison)
\end{itemize}

\section*{Procedure}
\begin{enumerate}
    \item Construct an electroscope: tape a vertical strip of foil inside a clear container and attach two thin foil strips at the bottom to act as leaves.
    \item Charge the electroscope negatively by rubbing a plastic comb with wool and touching it to the top foil.
    \item Observe that the leaves repel due to electrostatic force.
    \item Shine a visible flashlight on the top of the foil: no change should occur.
    \item Shine the UV flashlight on the foil: leaves should collapse as photoelectrons escape.
    \item Insert a pane of glass between the UV light and the foil: emission stops, proving UV is necessary.
\end{enumerate}

\section*{Extended Theory}

\subsection*{1. The Work Function and Photon Energy}
The \textbf{work function} $\phi$ is the minimum energy needed to remove an electron from a material's surface. It is typically expressed in electronvolts (eV).

\begin{itemize}
    \item Zinc: $\phi \approx 4.3\ \text{eV}$
    \item Aluminum: $\phi \approx 4.1\ \text{eV}$
\end{itemize}

The energy of a photon is:
\[
E = hf = \frac{hc}{\lambda}
\]

Where:
\begin{itemize}
    \item $h = 6.626 \times 10^{-34}$ J$\cdot$s
    \item $c = 3.00 \times 10^8$ m/s
    \item $\lambda$ is the wavelength (in meters)
\end{itemize}

In electronvolts:
\[
E (\text{eV}) = \frac{1240}{\lambda\ (\text{nm})}
\]

\textbf{Examples:}
\begin{itemize}
    \item $\lambda = 400$ nm: $E \approx 3.1$ eV
    \item $\lambda = 300$ nm: $E \approx 4.13$ eV
    \item $\lambda = 250$ nm: $E \approx 4.96$ eV
\end{itemize}

Thus, only UV light below ~300 nm consistently exceeds the work function for common metals.

\subsection*{2. Einstein's Evidence for $E = hf$}
Einstein proposed:
\[
K_{\text{max}} = hf - \phi
\]

Where:
\begin{itemize}
    \item $K_{\text{max}}$ is the kinetic energy of emitted electrons,
    \item $hf$ is photon energy,
    \item $\phi$ is the work function.
\end{itemize}

This explained:
\begin{itemize}
    \item Threshold frequency behavior
    \item Linear relationship of $K_{\text{max}}$ with frequency
    \item No delay in emission
    \item Intensity has no effect below threshold frequency
\end{itemize}

Robert Millikan confirmed this experimentally in 1916, measuring $K_{\text{max}}$ vs. $f$ and validating Einstein's model.

\subsection*{3. Electrostatics of the Foil Electroscope}
The repulsion between foil leaves is due to stored negative charge:
\[
Q = CV
\]
Where:
\begin{itemize}
    \item $Q$ is the charge,
    \item $C$ is capacitance (typically $1$--$10$ pF),
    \item $V$ is voltage (often ~100 V from comb charging)
\end{itemize}

\[
Q \approx 10^{-11}\ \text{C} \Rightarrow \sim 6 \times 10^7\ \text{electrons}
\]

If each UV photon ejects one electron, then:
\begin{itemize}
    \item Millions of photons must strike per second to significantly discharge the foil
    \item Visible photons (e.g., $\lambda > 400$ nm, $E < 3.1$ eV) cannot eject any electrons
\end{itemize}

\section*{Conclusion}
This experiment shows that:
\begin{itemize}
    \item Bright visible light cannot eject electrons
    \item Dim UV light can, immediately and efficiently
    \item UV-blocking glass stops emission
\end{itemize}

These observations support Einstein's photon theory and refute classical wave explanations.

\section*{Historical Context}
Einstein's 1905 paper, \emph{``Über einen die Erzeugung und Verwandlung des Lichtes betreffenden heuristischen Gesichtspunkt''}, proposed the revolutionary idea of light quanta. For this, he received the 1921 Nobel Prize in Physics:

\begin{quote}
    \emph{``For his services to theoretical physics, and especially for his discovery of the law of the photoelectric effect.''}
\end{quote}

\section*{References}
\begin{itemize}
    \item Einstein, A. (1905). \emph{Annalen der Physik}, 17, 132--148. \url{https://einsteinpapers.press.princeton.edu/vol2-trans/100}
    \item Feynman, R. (1963). \emph{The Feynman Lectures on Physics, Vol. 1, Chapter 37}
    \item Young, H. D., Freedman, R. A. (2019). \emph{University Physics}, 15th ed.
\end{itemize}

\end{document}
