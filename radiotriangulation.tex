% Radio Triangulation with Scratch & Python – Student Workbook (revised)
% -------------------------------------------------------------
\documentclass[11pt]{article}
\usepackage[margin=1in]{geometry}
\usepackage{amsmath,amssymb}
\usepackage{minted}   % requires --shell-escape or switch to listings
\usepackage{graphicx}
\usepackage{caption}  % <-- needed for \captionof
\usepackage{hyperref}
\hypersetup{colorlinks=true,linkcolor=blue,urlcolor=cyan}
\setlength{\parskip}{6pt}
\setlength{\parindent}{0pt}

\title{Radio Triangulation with Scratch~\&~Python\\\large A Guided Math and Coding Workbook}
\date{April 2025}

\begin{document}
\maketitle

% ------------------------------------------------------------------
%  Contents (kept on the first page – no forced page‑break afterwards)
% ------------------------------------------------------------------
\tableofcontents

% ================================================================
\section{Why Triangulation?}
Finding an emitter’s location from measurements at a distance is a classic applied‑math problem.  The same ideas drive GPS, seismic localization, and even wildlife‑tracking collars.  This workbook teaches the underlying geometry and linear‑algebra step by step, then shows how to implement each formula first in Scratch (for intuition) and finally in Python + NumPy (for power and realism).

% ================================================================
\section{Angle Basics and the Need for \texttt{atan2}}
When we convert Cartesian differences $(\Delta x,\,\Delta y)$ into an angle we want two things:
\begin{enumerate}
  \item an \emph{unambiguous} angle for all four quadrants and  
  \item continuity at the $\pm180^\circ$ wrap‑around.
\end{enumerate}
The one‑argument inverse tangent $\tan^{-1}(y/x)$ fails on both counts, while \texttt{atan2}(\textit{dy},\textit{dx}) succeeds by inspecting the signs of both arguments.  We formalise that idea in Equation~\eqref{eq:atan2} and will reference it throughout.

\begin{equation}
  \theta\;=\;\operatorname{atan2}(y,\,x)\,; \qquad -\pi < \theta \le \pi
  \label{eq:atan2}
\end{equation}
\noindent\textbf{Why this matters.}  By returning an angle on the open interval $(-\pi,\pi]$, Equation~\eqref{eq:atan2} produces a unique bearing no matter which quadrant the point $(x,y)$ lies in.  This eliminates the ambiguity that would otherwise plague our later algebra when we subtract bearings.

\begin{center}
  \includegraphics[width=.65\textwidth]{atan2_quadrants.png}
  \captionof{figure}{Quadrant handling by \texttt{atan2}.  The line under the diagram, $t=\frac{\rho}{2}=\dots$, is explained in Section~\ref{subsec:rhotheta}.}
  \label{fig:atan2again}
\end{center}

% -----------------------------------------------------------
\subsection{Interpreting the $t=\frac{\rho}{2}=\dots$ Annotation}
\label{subsec:rhotheta}
The extra formula sometimes shown beneath quadrant diagrams originates from polar‑to‑Cartesian conversion proofs.  Let $\rho$ be the radial distance and $\theta$ the polar angle.  Setting $t=\rho/2$ and rearranging $x=\rho\cos\theta$, $y=\rho\sin\theta$ one can show
\[
  t=-\theta\;\;\Longleftrightarrow\;\;(x,y)=\Bigl(-\tfrac{\rho}{t},\,\dots\Bigr),
\]
which is simply an algebraic trick for eliminating $\rho$.  It is \emph{not} something we will use operationally, but you may see it in textbooks; now you know where it comes from!

% ================================================================
\section{Two‑Receiver Geometry in Depth}
\subsection{Why Can We Set $\mathbf p_1=\mathbf p_2$?}
Each receiver’s line‑of‑bearing (LOB) is the set of all points that satisfy its parametric equation
\[\mathbf p_i(t_i)=\mathbf r_i+t_i\mathbf d_i.\]
If the transmitter lies on \emph{both} LOBs simultaneously, then it must satisfy both equations.  Therefore the true location $\mathbf p_{\text{true}}$ obeys
\[\mathbf p_{\text{true}}=\mathbf p_1(t_1)=\mathbf p_2(t_2).\]
Replacing $\mathbf p_{\text{true}}$ by our estimated $\hat{\mathbf p}$ gives us two linear equations in the two unknowns $t_1$ and $t_2$.  Solving those yields $\hat{\mathbf p}$.  The algebraic steps follow, and each one is annotated with its geometric meaning.

% -----------------------------------------------------------
\subsection{Step‑by‑Step Elimination}
\textbf{Start with component form.}  Writing out the $x$ and $y$ components explicitly we obtain
\begin{align}
  x_1 + t_1\cos\theta_1 &= x_2 + t_2\cos\theta_2,\label{eq:xcomp}\\
  y_1 + t_1\sin\theta_1 &= y_2 + t_2\sin\theta_2.\label{eq:ycomp}
\end{align}
\emph{Interpretation.}  Equation~\eqref{eq:xcomp} states that the unknown $x$‑coordinate of the transmitter has two alternative expressions—one from each receiver.  Equation~\eqref{eq:ycomp} says the same for the $y$‑coordinate.  By equating them we force the intersection.

\textbf{Eliminate $t_2$.}  Multiply Equation~\eqref{eq:xcomp} by $\sin\theta_2$ and Equation~\eqref{eq:ycomp} by $\cos\theta_2$ then subtract to eliminate $t_2$:
\begin{align}
  &(x_1-x_2)\sin\theta_2 + t_1(\cos\theta_1\sin\theta_2) \\
  &\quad-\bigl[(y_1-y_2)\cos\theta_2 + t_1(\sin\theta_1\cos\theta_2)\bigr] = 0.\label{eq:elim1}
\end{align}
\textbf{Use a trig identity.}  Factoring $t_1$ and applying the sine‑difference identity $\sin(\theta_2-\theta_1)=\sin\theta_2\cos\theta_1-\cos\theta_2\sin\theta_1$ simplifies Equation~\eqref{eq:elim1} to
\begin{equation}
  t_1\,\sin(\theta_2-\theta_1) = (x_2 - x_1)\,\sin\theta_2 - (y_2 - y_1)\,\cos\theta_2.
\end{equation}
Finally we isolate $t_1$ to obtain the celebrated intersection formula
\begin{equation}
  t_1\;=\;\frac{(x_2 - x_1)\,\sin\theta_2\;-\; (y_2 - y_1)\,\cos\theta_2}{\sin(\theta_2 - \theta_1)}.
  \label{eq:twoIntersection}
\end{equation}
\noindent\textbf{Why Equation~\eqref{eq:twoIntersection}?}  The numerator measures how far the two LOBs are offset from one another, projected into receiver~2’s reference frame; the denominator rescales that distance according to the angular separation between the LOBs.  A small denominator (nearly parallel LOBs) inflates $t_1$—in other words, a shallow intersection angle greatly amplifies positional uncertainty.

\subsection{Python Helper to Check Our Hand Algebra}
Using symbolic math (\textsf{SymPy}) we can verify Equation~\eqref{eq:twoIntersection} programmatically:
\begin{minted}[linenos,fontsize=\small]{python}
import sympy as sp
x1,x2,y1,y2,t1,t2 = sp.symbols('x1 x2 y1 y2 t1 t2', real=True)
th1, th2 = sp.symbols('th1 th2', real=True)
sol = sp.solve([
    x1 + t1*sp.cos(th1) - (x2 + t2*sp.cos(th2)),
    y1 + t1*sp.sin(th1) - (y2 + t2*sp.sin(th2))
], (t1,t2))
sp.simplify(sol[t1])  # matches Equation (t1)...
\end{minted}

% -----------------------------------------------------------
\subsection{Visualising the Geometry (Jupyter Cell)}
The following cell produces a plot of the two receivers, their LOBs, and the transmitter.  Run it inside your Jupyter notebook to “see” what the algebra is doing.
\begin{minted}[linenos,fontsize=\small]{python}
import numpy as np
import matplotlib.pyplot as plt

# receiver positions and true transmitter
r = np.array([[0,0], [400,0]])
p_true = np.array([180.0, 120.0])

# bearings (deg) derived from true geometry
th = np.degrees(np.arctan2(p_true[1]-r[:,1], p_true[0]-r[:,0]))

# build LOBs for plotting
L = 500  # plot length
lob_pts = []
for rx, angle in zip(r, th):
    direction = np.array([np.cos(np.radians(angle)), np.sin(np.radians(angle))])
    t = np.linspace(0, L, 100)[:,None]
    lob_pts.append(rx + t*direction)

plt.figure()
plt.plot(r[:,0], r[:,1], 'ko', label='Receivers')
plt.plot(p_true[0], p_true[1], 'r*', markersize=12, label='Transmitter')
for idx, line in enumerate(lob_pts):
    plt.plot(line[:,0], line[:,1], label=f'LOB {idx+1}')
plt.axis('equal')
plt.legend()
plt.title('Two-Receiver Triangulation Geometry')
plt.xlabel('x (m)')
plt.ylabel('y (m)')
plt.show()
\end{minted}
\noindent\emph{Interpretation.}  Notice how the transmitter lies exactly at the intersection of the two infinite LOBs, confirming the algebraic solution.

% ================================================================
\section{Three Receivers and Least Squares, Step By Step}
\subsection{Re‑using the Same Vector Form}
Keep each LOB in the familiar form
\[\mathbf p_i(t_i)=\mathbf r_i+t_i\mathbf d_i\qquad(i=1,2,3).\]
Writing out $x$ and $y$ components of all three yields six linear equations.  Rearranging gives Equation~\eqref{eq:matrix}.  We now dissect that equation line‑by‑line so you understand what every term means.

% -----------------------------------------------------------
\subsection{What Does $A\hat{\mathbf p}=\mathbf b$ Mean?}
\begin{itemize}
  \item $A$ is a $3\times2$ matrix whose rows contain the direction‑vector components.
  \item $\hat{\mathbf p}=\langle \hat x,\hat y\rangle^{\mathsf T}$ is the column vector we want.
  \item $\mathbf b$ encodes the known constants $d_{ix}x_i-d_{iy}y_i$.
\end{itemize}
If $A$ were square we could simply invert it, but with 3 equations vs.~2 unknowns we instead find the vector that minimises the residual error.  Linear‑algebra shows the solution is the \emph{normal equation}
\begin{equation}
  \hat{\mathbf p}\;=\;(A^{\mathsf T}A)^{-1}A^{\mathsf T}\mathbf b.
  \label{eq:LS}
\end{equation}
\noindent\textbf{Why minimising the squared error?}  Squared error penalises large misses more heavily than small ones and leads to a unique analytic solution for linear problems.  The projection interpretation also gives geometric intuition: we are projecting $\mathbf b$ onto $A$’s column space.

% -----------------------------------------------------------
\subsection{Walking Through $(A^{\mathsf T}A)$}
Take a concrete numerical example.
\begin{minted}[linenos,fontsize=\small]{python}
import numpy as np
# example bearings (degrees) and receiver coords
th = np.radians([ 10, 110, 250 ])
r  = np.array([[0,0], [400,0], [200,300]])
# build A and b
A = np.column_stack((np.cos(th), -np.sin(th)))
b = np.cos(th)*r[:,0] - np.sin(th)*r[:,1]
# Manual multiplication
AtA = A.T @ A          # 2x2
Atb = A.T @ b          # 2x1
print('A^T A =\n',AtA)
print('A^T b =',Atb)
\end{minted}
Explain each product:
\[
  (A^{\mathsf T}A)_{11}=\sum_{i=1}^3 d_{ix}^2,\; (A^{\mathsf T}A)_{12}=\sum_{i=1}^3 (-d_{ix}d_{iy}),\;\text{etc.}
\]
Because $A^{\mathsf T}A$ is symmetric we only need compute its upper triangle manually.

% -----------------------------------------------------------
\subsection{Finding the Inverse by Hand (2×2 Case)}
For a $2\times2$ matrix $M=\begin{bmatrix}a&b\\b&c\end{bmatrix}$
\[
  M^{-1}=\frac{1}{ac-b^2}\begin{bmatrix}c&-b\\-b&a\end{bmatrix}.
\]
NumPy does this for us, but it is instructive to check the determinant $ac-b^2$ is non‑zero, which requires the three bearings not to be co‑linear.

\begin{minted}[linenos,fontsize=\small]{python}
det = AtA[0,0]*AtA[1,1] - AtA[0,1]**2
assert det != 0, 'Geometry is degenerate!'
inv = np.linalg.inv(AtA)
print('inv(AtA)=\n',inv)
# finally p_hat
p_hat = inv @ Atb
print('Estimated transmitter =',p_hat)
\end{minted}

\subsection{Why Does This Work?}
The normal‑equation solution arises from setting the gradient of the squared error $\lVert A\mathbf p-\mathbf b\rVert^2$ with respect to $\mathbf p$ to zero, which yields $A^{\mathsf T}A\mathbf p=A^{\mathsf T}\mathbf b$.  That minimum‑error vector is exactly Equation~\eqref{eq:LS}.  You can read any linear‑algebra text for proof, but the intuition is we are projecting $\mathbf b$ onto the column space of $A$.

% -----------------------------------------------------------
\subsection{Covariance Revisited}
Plugging $M=(A^{\mathsf T}A)^{-1}$ into
\[
  \operatorname{Cov}(\hat{\mathbf p}) = \sigma^2 M
\]
reveals how the variances scale with geometry.  In Python:
\begin{minted}[fontsize=\small]{python}
# assume sigma = 3° measurement noise (in radians)
sigma = np.deg2rad(3)
Cov = sigma**2 * inv
print('Position covariance matrix:\n', Cov)
print('Std‑dev in x, y =', np.sqrt(np.diag(Cov)))
\end{minted}
Interpret those numbers: one standard deviation encloses $\approx68\%$ of true transmitter locations under repeated noise trials.

% ================================================================
\section{Putting It All Together}
\begin{enumerate}
  \item Build the Scratch two‑receiver demo; verify the lines intersect visually.
  \item Run the Python script with randomised noise to see statistical scatter.
  \item Add or move receivers; observe how $A^{\mathsf T}A$ and the covariance change.
\end{enumerate}

% -----------------------------------------------------------
\section*{Appendix A – Full Python Listing}
\begin{minted}[fontsize=\small]{python}
"""Full three‑receiver least‑squares demo."""
import numpy as np, math, random
# receiver positions
r = np.array([[0,0], [400,0], [200,300]], float)
# true transmitter (unknown to algorithm)
p_true = np.array([180.0, 120.0])
# simulate noisy bearings
sigma_deg = 3.0
th_true = np.degrees(np.arctan2(p_true[1]-r[:,1], p_true[0]-r[:,0]))
noise = np.random.normal(0, sigma_deg, size=3)
th_meas = np.radians(th_true + noise)
# build A, b
A = np.column_stack((np.cos(th_meas), -np.sin(th_meas)))
b = np.cos(th_meas)*r[:,0] - np.sin(th_meas)*r[:,1]
# solve
p_hat = np.linalg.inv(A.T @ A) @ (A.T @ b)
print('True :', p_true, '\nEst. :', p_hat)
print('Error:', np.linalg.norm(p_hat-p_true), 'pixels')
\end{minted}

% -----------------------------------------------------------
\vfill\noindent\small© 2025 You. Licensed CC BY‑SA 4.0.

\end{document}

